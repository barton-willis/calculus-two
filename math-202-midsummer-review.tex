\documentclass[12pt,fleqn]{exam}
\usepackage{pifont}
\usepackage{dingbat}
\usepackage{amsmath,amssymb}
\usepackage{epsfig}
\usepackage[]{hyperref}
\usepackage{geometry}
\geometry{letterpaper, margin=0.75in}
\addpoints
\boxedpoints
\pointsinmargin
\pointname{pts}

\usepackage[activate={true,nocompatibility},final,tracking=true,kerning=true,factor=1100,stretch=10,shrink=10]{microtype}
\usepackage[american]{babel}
%\usepackage[T1]{fontenc}
\usepackage{fourier}
\usepackage{isomath}
\usepackage{upgreek,amsmath}
\usepackage{amssymb}

\newcommand{\dotprod}{\, {\scriptzcriptztyle
    \stackrel{\bullet}{{}}}\,}

\newcommand{\reals}{\mathbf{R}}
\newcommand{\lub}{\mathrm{lub}} 
\newcommand{\glb}{\mathrm{glb}} 
\newcommand{\complex}{\mathbf{C}}
\newcommand{\dom}{\mbox{dom}}
\newcommand{\cover}{{\mathcal C}}
\newcommand{\integers}{\mathbf{Z}}
\newcommand{\vi}{\, \mathbf{i}}
\newcommand{\vj}{\, \mathbf{j}}
\newcommand{\vk}{\, \mathbf{k}}
\newcommand{\bi}{\, \mathbf{i}}
\newcommand{\bj}{\, \mathbf{j}}
\newcommand{\bk}{\, \mathbf{k}}
\DeclareMathOperator{\Arg}{\mathrm{Arg}}
\DeclareMathOperator{\Ln}{\mathrm{Ln}}
\newcommand{\imag}{\, \mathrm{i}}
\newcommand{\range}{\mathrm{range}}
\newcommand{\ball}{\mathrm{ball}}
\newcommand{\LP}{\mathrm{LP}}

\usepackage{graphicx}
\newcommand\AM{{\sc am}}
\newcommand\PM{{\sc pm}}
     
\newcommand{\quiz}{0}
\newcommand{\term}{Fall}
\newcommand{\due}{Friday 25 August  at 11:59 \PM}
\begin{document}
\large
\vspace{0.1in}
\noindent\makebox[3.0truein][l]{{\bf MATH 202}}
{\bf Name:} \hrulefill \\
\noindent \makebox[3.0truein][l]{\bf Calculus Practice I, \term \/ \the\year}
%{\bf Row:}\hrulefill\
\vspace{0.1in}

\noindent Here is an opportunity for you to maintain your calculus skills
over the summer. If you complete these problems,
digitize your work, and submit your work to Canvas, I will send you my
solutions. If you need some help with these questions, 
email me with your questions (\href{mailto:willisb@unk.edu}{willisb@unk.edu})

Completing this work is optional, and it does not 
enter into your class grade in any way--this work is 
 not a bonus, extra credit, or anything like that.

\begin{questions}


\question Find an equation of the tangent line to the 
curve $y = \sqrt{x^2+1}$ at the point \mbox{$(x=1, y=\sqrt{2})$.}
\begin{solution}[2.5in]
\end{solution}

\newpage

\question Find each antiderivative.

\begin{parts}
    
    \part $\int x^2 - x - 2 \, \mathrm{d} x $
    \begin{solution}[2.5in]
    \end{solution}

    \part $\int (x-1)(x+2) \, \mathrm{d} x $
    \begin{solution}[2.5in]
    \end{solution}



    \part $\int \frac{1+x^2}{x^2} \, \mathrm{d} x$


\end{parts}

\newpage

\question Find each definite integral.


\begin{parts}
    
    \part $\int_1^2 x^2 - x - 2 \, \mathrm{d} x $
    \begin{solution}[2.5in]
    \end{solution}

    \part $\int_1^2 (x-1)(x+2) \, \mathrm{d} x $
    \begin{solution}[2.5in]
    \end{solution}



    \part $\int_1^4 \frac{1+x^2}{x^2} \, \mathrm{d} x$


\end{parts}

\newpage

\question For each function $F$, find the solution set of $F^\prime(x)  = 0$.  

\begin{parts}

    \part $F(x) = x^2+x+3$
    \begin{solution}[2.5in]
    \end{solution}

    \part $F(x) = (x-3)(x^2+3)$
    \begin{solution}[2.5in]
    \end{solution}

    \part $F(x) = 2 x +  \frac{x}{x-2}$
    \begin{solution}[2.5in]
    \end{solution}

    \part $F(x) = \cos(x) \sin(x)$
    \begin{solution}[2.5in]
    \end{solution}


\end{parts}

\newpage

\question Find the value of each limit.

\begin{parts}

\part $\displaystyle \lim_{x \to 4} \frac{\cos(x)+1}{x-3}$
\begin{solution}[2.5in]
    
\end{solution}

\part $\displaystyle \lim_{x \to 4} \frac{{{x}^{2}}-2 x-3}{x-3}$
\begin{solution}[2.5in]
    
\end{solution}

\part $\displaystyle \lim_{x \to \infty} \frac{5 x^2 + x+1}{7 x^2 + 107}$

\end{parts}
\end{questions}




\end{document}