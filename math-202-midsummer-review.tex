\documentclass[12pt,fleqn,answers]{exam}
\usepackage{pifont}
\usepackage{dingbat}
\usepackage{amsmath,amssymb}
\usepackage{epsfig}
\usepackage{graphicx}
\usepackage[]{hyperref}
\usepackage{geometry}
\geometry{letterpaper, margin=0.75in}
\addpoints
\boxedpoints
\pointsinmargin
\pointname{pts}

\usepackage[activate={true,nocompatibility},final,tracking=true,kerning=true,factor=1100,stretch=10,shrink=10]{microtype}
\usepackage[american]{babel}
%\usepackage[T1]{fontenc}
\usepackage{fourier}
\usepackage{isomath}
\usepackage{upgreek,amsmath}
\usepackage{amssymb}

\newcommand{\dotprod}{\, {\scriptzcriptztyle
    \stackrel{\bullet}{{}}}\,}

\newcommand{\reals}{\mathbf{R}}
\newcommand{\lub}{\mathrm{lub}} 
\newcommand{\glb}{\mathrm{glb}} 
\newcommand{\complex}{\mathbf{C}}
\newcommand{\dom}{\mbox{dom}}
\newcommand{\cover}{{\mathcal C}}
\newcommand{\integers}{\mathbf{Z}}
\newcommand{\vi}{\, \mathbf{i}}
\newcommand{\vj}{\, \mathbf{j}}
\newcommand{\vk}{\, \mathbf{k}}
\newcommand{\bi}{\, \mathbf{i}}
\newcommand{\bj}{\, \mathbf{j}}
\newcommand{\bk}{\, \mathbf{k}}
\DeclareMathOperator{\Arg}{\mathrm{Arg}}
\DeclareMathOperator{\Ln}{\mathrm{Ln}}
\newcommand{\imag}{\, \mathrm{i}}
\newcommand{\range}{\mathrm{range}}
\newcommand{\ball}{\mathrm{ball}}
\newcommand{\LP}{\mathrm{LP}}

\usepackage{graphicx}
\newcommand\AM{{\sc am}}
\newcommand\PM{{\sc pm}}
     
\newcommand{\quiz}{0}
\newcommand{\term}{Fall}
\newcommand{\due}{Friday 25 August  at 11:59 \PM}
\begin{document}
\large
\vspace{0.1in}
\noindent\makebox[3.0truein][l]{{\bf MATH 202}}
{\bf Name:} \hrulefill \\
\noindent \makebox[3.0truein][l]{\bf Calculus Practice I, \term \/ \the\year}
%{\bf Row:}\hrulefill\
\vspace{0.1in}

\noindent Here is an opportunity for you to maintain your calculus skills
over the summer. If you complete these problems,
digitize your work, and submit your work to Canvas, I will send you my
solutions. If you need some help with these questions, 
email me with your questions (\href{mailto:willisb@unk.edu}{willisb@unk.edu})

Completing this work is optional, and it does not 
enter into your class grade in any way--this work is 
 not a bonus, extra credit, or anything like that.

\begin{questions}


\question Find an equation of the tangent line to the 
curve $y = \sqrt{x^2+1}$ at the point \mbox{$(x=1, y=\sqrt{2})$.}
\begin{solution}[2.5in]

To find an equation of a line we need to know (a) its slope and (b) a point on the
line.  We're given a point on the line, so our main task is to find the slope of the 
tangent line. To do that we need to first find a formula for \(\displaystyle
\frac{\mathrm{d} y}{\mathrm{d} x} \) and second we need to evaluate \(\displaystyle
\frac{\mathrm{d} y}{\mathrm{d} x} \) when $x=1$. The chain rule tell us that
\begin{equation}
    \frac{\mathrm{d} y}{\mathrm{d} x} = \frac{1}{2} \times 2 x (x^2+1)^{-1/2}
\end{equation}
And pasting in $x \to 1$, we have
\begin{equation}
    \left . \frac{\mathrm{d} y}{\mathrm{d} x} \right |_{x=1}=  
    \left . \frac{1}{2} \times 2 x (x^2+1)^{-1/2} \right |_{x=1} =
    \frac{1}{\sqrt{2}}.
\end{equation}
An equation of the given tangent line is
\begin{equation}
  y - \sqrt{2} = \frac{1}{\sqrt{2}} (x-1)
\end{equation}
Traditionally, we would simplify \(\frac{1}{\sqrt{2}}\)
to \(\frac{\sqrt{2}}{2}\). But in this context, I 
don't see an advantage to doing this. So let us LIB 
(let it be).

As a check to our work, let's ask Desmos to draw 
graphs of both $y=\sqrt{1+x^2}$ and $y - \sqrt{2} = \frac{1}{\sqrt{2}} (x-1)$. Does the line and the curve
appear to be tangent?  Sure.

    \centering
    \includegraphics[scale=0.2]{desmos-graph(44).png}

    


\end{solution}

\newpage

\question Find each antiderivative.

\begin{parts}
    
    \part $\int x^2 - x - 2 \, \mathrm{d} x $
    \begin{solution}[2.5in]
  
    \begin{equation}
      \int x^2 - x - 2 \, \mathrm{d} x = \frac{1}{3} x^3 - \frac{1}{2} x^2 - 2 x.
      \end{equation}

    Some teachers will insist on the $+c$.  I say let's just
    remember that all antiderivatives are undetermined 
    up to an additive constant and for forget the silly $+c$ rule.
    \end{solution}

    \part $\int (x-1)(x+2) \, \mathrm{d} x $
    \begin{solution}[2.5in]
    \end{solution}



    \part $\int \frac{1+x^2}{x^2} \, \mathrm{d} x$

\begin{solution} Expanding the integrand shows that this problem is identical to the previous.

\end{solution}
\end{parts}

\newpage

\question Find each definite integral.


\begin{parts}
    
    \part $\int_1^2 x^2 - x - 2 \, \mathrm{d} x $
    \begin{solution}[2.5in]
    In the previous question, we found the antiderivative; so all we need to do is
    \begin{equation}
    \int_1^2 x^2 - x - 2 \, \mathrm{d} x = \left . \frac{1}{3} x^3 - \frac{1}{2} x^2 - 2 x \right |_{x=1}^{x=2}
     = - \frac{7}{6}.
    \end{equation}
    Graphically, we can tell that this definite integral is negative. Why? The graph of the integrand is entirely below
    the x-axis, that's why. Here is a graph of the integrand
    
      \centering
    \includegraphics[scale=0.2]{desmos-graph(45).png}
    
    Actually, the graph of the integrand on the interval $[1,2]$ is pretty well approximated by a line segment joining 
    $(x=1,y=-2)$ and  $(x=2,y=0)$. Doing so, we see that  $ \int_1^2 x^2 - x - 2 \, \mathrm{d} x $ is pretty close
    to the negative of the area of a triangle with base one and height two.
    

    \end{solution}

    \part $\int_1^2 (x-1)(x+2) \, \mathrm{d} x $
    \begin{solution}[2.5in]
    Expanding the integrand, we see that we've done this problem before.
    \end{solution}



    \part $\int_1^4 \frac{1+x^2}{x^2} \, \mathrm{d} x$


\end{parts}

\newpage

\question For each function $F$, find the solution set of $F^\prime(x)  = 0$.  

\begin{parts}

    \part $F(x) = x^2+x+3$
    \begin{solution}[2.5in]
    \end{solution}

    \part $F(x) = (x-3)(x^2+3)$
    \begin{solution}[2.5in]
    \end{solution}

    \part $F(x) = 2 x +  \frac{x}{x-2}$
    \begin{solution}[2.5in]
    \end{solution}

    \part $F(x) = \cos(x) \sin(x)$
    \begin{solution}[2.5in]
    \end{solution}


\end{parts}

\newpage

\question Find the value of each limit.

\begin{parts}

\part $\displaystyle \lim_{x \to 4} \frac{\cos(x)+1}{x-3}$
\begin{solution}[2.5in] The function is continuous at 4, so we can use DS (direct substitution); we have
\begin{equation}
 \lim_{x \to 4} \frac{\cos(x)+1}{x-3} = \cos(4)+1.
\end{equation}
If somebody wants a decimal approximation to $\cos(4)+1$ to any number of digits, they can do that.
    
\end{solution}

\part $\displaystyle \lim_{x \to 4} \frac{{{x}^{2}}-2 x-3}{x-3}$
\begin{solution}[2.5in] Again, the function is continuous at 4, so let's use DS; we have
\begin{equation}
  \lim_{x \to 4} \frac{{{x}^{2}}-2 x-3}{x-3} = 4^2-2 \times 4 - 3 = 5.
\end{equation}
    
\end{solution}

\part $\displaystyle \lim_{x \to \infty} \frac{5 x^2 + x+1}{7 x^2 + 107}$

\end{parts}
\end{questions}




\end{document}