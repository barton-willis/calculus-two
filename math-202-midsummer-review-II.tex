\documentclass[12pt,fleqn,answers]{exam}
\usepackage{pifont}
\usepackage{dingbat}
\usepackage{amssymb}
\usepackage{epsfig}
\usepackage{graphicx}
\usepackage[]{hyperref}
\usepackage{geometry}
\usepackage[intlimits]{amsmath}
\geometry{letterpaper, margin=0.75in}
\addpoints
\boxedpoints
\pointsinmargin
\pointname{pts}

\usepackage[activate={true,nocompatibility},final,tracking=true,kerning=true,factor=1100,stretch=10,shrink=10]{microtype}
\usepackage[american]{babel}
%\usepackage[T1]{fontenc}
\usepackage{fourier}
\usepackage{isomath}
\usepackage{upgreek,amsmath}
\usepackage{amssymb}

\newcommand{\dotprod}{\, {\scriptzcriptztyle
    \stackrel{\bullet}{{}}}\,}

\newcommand{\reals}{\mathbf{R}}
\newcommand{\lub}{\mathrm{lub}} 
\newcommand{\glb}{\mathrm{glb}} 
\newcommand{\complex}{\mathbf{C}}
\newcommand{\dom}{\mbox{dom}}
\newcommand{\cover}{{\mathcal C}}
\newcommand{\integers}{\mathbf{Z}}
\newcommand{\vi}{\, \mathbf{i}}
\newcommand{\vj}{\, \mathbf{j}}
\newcommand{\vk}{\, \mathbf{k}}
\newcommand{\bi}{\, \mathbf{i}}
\newcommand{\bj}{\, \mathbf{j}}
\newcommand{\bk}{\, \mathbf{k}}
\DeclareMathOperator{\Arg}{\mathrm{Arg}}
\DeclareMathOperator{\Ln}{\mathrm{Ln}}
\newcommand{\imag}{\, \mathrm{i}}
\newcommand{\range}{\mathrm{range}}
\newcommand{\ball}{\mathrm{ball}}
\newcommand{\LP}{\mathrm{LP}}

\usepackage{graphicx}
\newcommand\AM{{\sc am}}
\newcommand\PM{{\sc pm}}
     
\newcommand{\quiz}{0}
\newcommand{\term}{Fall}
\newcommand{\due}{Friday 25 August  at 11:59 \PM}
\begin{document}
\large
\vspace{0.1in}
\noindent\makebox[3.0truein][l]{{\bf MATH 202}}
{\bf Name:} \hrulefill \\
\noindent \makebox[3.0truein][l]{\bf Calculus Practice II, \term \/ \the\year}
%{\bf Row:}\hrulefill\
\vspace{0.1in}

\noindent Here is an opportunity for you to maintain your calculus skills
over the summer. If you complete these problems,
digitize your work, and submit your work to Canvas, I will send you my
solutions. If you need some help with these questions, 
email me with your questions (\href{mailto:willisb@unk.edu}{willisb@unk.edu})

Completing this work is optional, and it does not 
enter into your class grade in any way--this work is 
 not a bonus, extra credit, or anything like that.

\begin{questions}


\question Use a substitution to find each antiderivative.

\begin{parts}

\part $\int x \cos(x^2) \, \mathrm{d} x $
\begin{solution}[1.5in]
Choose $z = x^2$. Then $\mathrm{d} z = 2 x \mathrm{d} x$. So 
\begin{equation*}
    \int x \cos(x^2) \, \mathrm{d} x = \int \frac{1}{2} \cos(z) \, \mathrm{d} z =
    \frac{1}{2} \sin(z) =  \frac{1}{2} \sin(x^2).
\end{equation*}
We should develop the habit of checking antiderivatives by differentiation; we 
have
\begin{equation*}
    \frac{\mathrm{d}}{\mathrm{d} x} \left( \frac{1}{2} \sin(x^2) \right)
      = \frac{1}{2} \times 2 x  \cos(x^2) = x  \cos(x^2).
\end{equation*}

    
\end{solution}

\part $\int \frac{x}{1+x^2} \, \mathrm{d} x $
\begin{solution}[1.5in]
Choose $z = 1+x^2$. Then $\mathrm{d}z = 2 x \mathrm{d} x$. So 
\begin{equation*}
    \int \frac{x}{1+x^2} \, \mathrm{d} x =
     \int \frac{1}{2 z} \, \mathrm{d} z =
    \frac{1}{2} \ln(z) =  \frac{1}{2} \ln(x^2+1).
\end{equation*}
    
\end{solution}

\part $\int x |x^2 - 1| \, \mathrm{d} x $
\begin{solution}[1.5in]
Choose $z = x^2-1$. So $\mathrm{d} z = 2 x \mathrm{d} x$. So 
\begin{equation*}
    \int x |x^2 - 1| \, \mathrm{d} x 
    = \int \frac{1}{2} |z| \, \mathrm{d} z 
    = \frac{1}{4} z |z|
    = \frac{1}{4} \left(x^2 - 1\right) |x^2 - 1|
\end{equation*}
Here we used the not so well known fact that 
\begin{equation*}
    \int |x| \, \mathrm{d} x = \frac{1}{2} x |x|.
\end{equation*}

    
\end{solution}

\part $\int x^2 \sqrt{x^3+1} \, \mathrm{d} x $
\begin{solution}[1.5in]
Choose $z = x^3+1$. Then $\mathrm{d}z = 3 x^2 \mathrm{d} x$. So 
\begin{equation*}
    \int x^2 \sqrt{x^3+1} \, \mathrm{d} x 
    = \int \frac{1}{3} \sqrt{z} \, \mathrm{d} z 
    = \frac{2 {{z}^{\frac{3}{2}}}}{9}
    = \frac{2 {{\left( {{x}^{3}}+1\right) }^{\frac{3}{2}}}}{9}.
\end{equation*}

    
\end{solution}

\end{parts}
\end{questions}




\end{document}