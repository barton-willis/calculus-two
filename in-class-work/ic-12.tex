\documentclass[12pt,fleqn]{exam}
\usepackage{amssymb}
\usepackage[intlimits]{amsmath}
\usepackage{epsfig}
\usepackage{upgreek}
\usepackage[super]{nth}
\usepackage[colorlinks=true,linkcolor=black,anchorcolor=black,citecolor=black,filecolor=black,menucolor=black,runcolor=black,urlcolor=black]{hyperref}
\usepackage[letterpaper, margin=0.75in]{geometry}
\addpoints
\boxedpoints
\pointsinmargin
\pointname{pts}
\usepackage{tikz}
\usepackage{tkz-euclide}
\usetikzlibrary{shapes.geometric}
\usetikzlibrary{calc}
\usepackage[final]{microtype}
\frenchspacing
\usepackage[american]{babel}
\usepackage[T1]{fontenc}
\usepackage[]{fourier}
\usepackage{isomath}
\usepackage{upgreek,amsmath}
\usepackage{graphicx}

\newcommand{\dotprod}{\, {\scriptzcriptztyle\stackrel{\bullet}{{}}}\,}

\newcommand{\reals}{\mathbf{R}}
\newcommand{\lub}{\mathrm{lub}} 
\newcommand{\glb}{\mathrm{glb}} 
\newcommand{\complex}{\mathbf{C}}
\newcommand{\dom}{\mbox{dom}}
\newcommand{\range}{\mbox{range}}
\newcommand{\cover}{{\mathcal C}}
\newcommand{\integers}{\mathbf{Z}}
\newcommand{\vi}{\, \mathbf{i}}
\newcommand{\vj}{\, \mathbf{j}}
\newcommand{\vk}{\, \mathbf{k}}
\newcommand{\bi}{\, \mathbf{i}}
\newcommand{\bj}{\, \mathbf{j}}
\newcommand{\bk}{\, \mathbf{k}}
\DeclareMathOperator{\Arg}{\mathrm{Arg}}
\DeclareMathOperator{\Ln}{\mathrm{Ln}}
\newcommand{\imag}{\, \mathrm{i}}

\usepackage{graphicx}
\usepackage{color}
%\shadedsolutions
%\definecolor{SolutionColor}{rgb}{1,0.72,0.46} %{0.8,0.9,1}
\newcommand\AM{\textsc{am}}
\newcommand\PM{\textsc{pm}}
     
\newcommand{\quiz}{12}
\newcommand{\term}{Fall}
\newcommand{\due}{Thursday 5 October 13:20}
\newcommand{\class}{MATH 202, Fall \the\year}
\begin{document}
\large
\vspace{0.1in}
\noindent\makebox[3.0truein][l]{\textbf{\class}}
\textbf{Name:} \hrulefill \\
\noindent \makebox[3.0truein][l]{\textbf{In class work  \quiz}}
\textbf{Row and Seat}:\hrulefill\\

\begin{quote}
    \emph{“Study hard what interests you the most in the most 
    undisciplined, irreverent and original manner possible.”} 
    \hfill \sc{Richard Feynman} 
\end{quote}

\noindent  In class work  \textbf{\quiz}  has questions \textbf{1} 
through  \textbf{\numquestions} \/ with a total of 
\textbf{\numpoints\/}  points.Turn in your work at the end of class 
\emph{on paper}. This assignment is due \emph{\due}.

\vspace{0.1in}


\begin{questions} 
    
  \question[2] Find the \emph{numeric value} of the integral 
  $\int_0^\infty \frac{x}{1+x^4} \, \mathrm{d} x$.
  \textbf{Hint:} To find an antiderivative of $\int \frac{x}{1+x^4} \, \mathrm{d} x$, use 
  the substitution $z = x^2$.

  \begin{solution}[3.0in]
  \end{solution}


  \question [1] Show that $\int_0^\infty \frac{28 + \cos(x)}{1+x^2} \, \mathrm{d} x$
  converges. To do this, use a comparison test with 
  $ \frac{\alpha}{1+x^2}$, where $\alpha$ is a number that you cleverly 
  choose.
  \begin{solution}%[3.0in]
  \end{solution}

  \newpage
  \question [1] Show that 
  $\int_1^\infty \frac{107 + \mathrm{e}^{-x}}{1+x^2} \, \mathrm{d} x$
  converges. To do this, use a limit comparison test. 
  \begin{solution}[3.0in]
  \end{solution}

  \end{questions}
\end{document}

