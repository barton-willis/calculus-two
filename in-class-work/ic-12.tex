\documentclass[12pt,fleqn,answers]{exam}
\usepackage{amssymb}
\usepackage[intlimits]{amsmath}
\usepackage{epsfig}
\usepackage{upgreek}
\usepackage[super]{nth}
\usepackage[colorlinks=true,linkcolor=black,anchorcolor=black,citecolor=black,filecolor=black,menucolor=black,runcolor=black,urlcolor=black]{hyperref}
\usepackage[letterpaper, margin=0.75in]{geometry}
\addpoints
\boxedpoints
\pointsinmargin
\pointname{pts}
\usepackage{tikz}
\usepackage{tkz-euclide}
\usetikzlibrary{shapes.geometric}
\usetikzlibrary{calc}
\usepackage[final]{microtype}
\frenchspacing
\usepackage[american]{babel}
\usepackage[T1]{fontenc}
\usepackage[]{fourier}
\usepackage{isomath}
\usepackage{upgreek,amsmath}
\usepackage{graphicx}

\newcommand{\dotprod}{\, {\scriptzcriptztyle\stackrel{\bullet}{{}}}\,}

\newcommand{\reals}{\mathbf{R}}
\newcommand{\lub}{\mathrm{lub}} 
\newcommand{\glb}{\mathrm{glb}} 
\newcommand{\complex}{\mathbf{C}}
\newcommand{\dom}{\mbox{dom}}
\newcommand{\range}{\mbox{range}}
\newcommand{\cover}{{\mathcal C}}
\newcommand{\integers}{\mathbf{Z}}
\newcommand{\vi}{\, \mathbf{i}}
\newcommand{\vj}{\, \mathbf{j}}
\newcommand{\vk}{\, \mathbf{k}}
\newcommand{\bi}{\, \mathbf{i}}
\newcommand{\bj}{\, \mathbf{j}}
\newcommand{\bk}{\, \mathbf{k}}
\DeclareMathOperator{\Arg}{\mathrm{Arg}}
\DeclareMathOperator{\Ln}{\mathrm{Ln}}
\newcommand{\imag}{\, \mathrm{i}}

\usepackage{graphicx}
\usepackage{color}
%\shadedsolutions
%\definecolor{SolutionColor}{rgb}{1,0.72,0.46} %{0.8,0.9,1}
\newcommand\AM{\textsc{am}}
\newcommand\PM{\textsc{pm}}
     
\newcommand{\quiz}{12}
\newcommand{\term}{Fall}
\newcommand{\due}{Thursday 5 October 13:20}
\newcommand{\class}{MATH 202, Fall \the\year}
\begin{document}
\large
\vspace{0.1in}
\noindent\makebox[3.0truein][l]{\textbf{\class}}
\textbf{Name:} \hrulefill \\
\noindent \makebox[3.0truein][l]{\textbf{In class work  \quiz}}
\textbf{Row and Seat}:\hrulefill\\

\begin{quote}
    \emph{“Study hard what interests you the most in the most 
    undisciplined, irreverent and original manner possible.”} 
    \hfill \sc{Richard Feynman} 
\end{quote}

\noindent  In class work  \textbf{\quiz}  has questions \textbf{1} 
through  \textbf{\numquestions} \/ with a total of 
\textbf{\numpoints\/}  points.Turn in your work at the end of class 
\emph{on paper}. This assignment is due \emph{\due}.

\vspace{0.1in}


\begin{questions} 
    
  \question[2] Find the \emph{numeric value} of the integral 
  $\int_0^\infty \frac{x}{1+x^4} \, \mathrm{d} x$.
  \textbf{Hint:} To find an antiderivative of $\int \frac{x}{1+x^4} \, \mathrm{d} x$, use 
  the substitution $z = x^2$.

  \begin{solution}[3.0in] Let's begin by finding an antiderivative; once
    we found it, we'll use the FTC along with a limit to find the value 
    of the improper integral. We have
    \begin{align*}
    \int \frac{x}{1+x^4} \, \mathrm{d} x &= \frac{1}{2}  \int \, \frac{1}{1+(x^2)^2} \, \mathrm{d} x^2,  
                          && (\frac{1}{2} \mathrm{d} x^2 = 2 x \mathrm{d} x) \\
                                                         &= \frac{1}{2}  \int  \frac{1}{1+z^2} \, \mathrm{d} z, \\
                                                         &= \frac{1}{2} \arctan(z), \\
                                                          &= \frac{1}{2} \arctan(x^2).
    \end{align*}
    Second, we take on the improper integral:
     \begin{align*}
    \int_0^\infty  \frac{x}{1+x^4} \, \mathrm{d} x &= \lim_{a \to \infty}  \int_0^a  \frac{x}{1+x^4} \, \mathrm{d} x, \\
                &=  \lim_{a \to \infty} \left( \frac{1}{2} \arctan(x^2) \right |_{0}^a, \\
                &=  \lim_{a \to \infty} \left( \frac{1}{2} \arctan(a^2)  -  \frac{1}{2} \arctan(0) \right), \\
                 &=  \lim_{a \to \infty} \left( \frac{1}{2} \arctan(a^2)  \right), \\
                 &= \frac{\uppi}{4}
    \end{align*}
  \end{solution}


  \question [1] Show that $\int_0^\infty \frac{28 + \cos(x)}{1+x^2} \, \mathrm{d} x$
  converges. To do this, use a comparison test with 
  $ \frac{\alpha}{1+x^2}$, where $\alpha$ is a number that you cleverly 
  choose.
  \begin{solution} For all real numbers $x$, we have $ 27 \leq 28 + \cos(x) \leq 29$. Let's (cleverly) choose $\alpha$
  to be $29$. Then for all  real numbers $x$, we have
  \begin{equation}
    0 \leq  \frac{28 + \cos(x)}{1+x^2} \leq  \frac{29}{1+x^2}.
  \end{equation}
  But $\int_0^\infty \frac{29}{1+x^2} \, \mathrm{d} x$ converges, so $\int_0^\infty \frac{28 + \cos(x)}{1+x^2} \, \mathrm{d} x$ converges.
  
  \textbf{Be careful} We only know that $\int_0^\infty \frac{28 + \cos(x)}{1+x^2} \, \mathrm{d} x$ is a real number,
  but the comparison test \textbf{doesn't} tell us its value.  We'll it does tell us that
  \begin{equation*}
    \int_0^\infty \frac{28 + \cos(x)}{1+x^2} \, \mathrm{d} x  \leq  \int_0^\infty \frac{29}{1+x^2} \, \mathrm{d} x
    = \frac{29 \ensuremath{\pi} }{2} \approx 45.553093477052.
  \end{equation*}
  Numerical integration gives us the approximation $ \int_0^\infty \frac{28 + \cos(x)}{1+x^2} \, \mathrm{d} x \approx
   44.560$
  \end{solution}

  \newpage
  \question [1] Show that 
  $\int_1^\infty \frac{107 + \mathrm{e}^{-x}}{1+x^2} \, \mathrm{d} x$
  converges. To do this, use a limit comparison test. 
  \begin{solution}[3.0in] We know that  $\int_1^\infty \frac{1}{1+x^2} \, \mathrm{d} x$ converges. And
  for all real $x \geq 1$ we have $\frac{107 + \mathrm{e}^{-x}}{1+x^2} > 0$ and $\frac{1}{1+x^2} \geq 0$.
  Finally, everything in sight is continuous; so look at
  \begin{align*}
   \lim_{x \to \infty} \frac{\frac{1}{1+x^2}} {\frac{107 + \mathrm{e}^{-x}} {1+x^2} } &=
    \lim_{x \to \infty} \frac{1}{107 + \mathrm{e}^{-x}}, \\
    &= \frac{1}{107}.   
    \end{align*}
  So $\int_1^\infty \frac{107 + \mathrm{e}^{-x}}{1+x^2} \, \mathrm{d} x$ converges.
  \end{solution}

  \end{questions}
\end{document}

