\documentclass[12pt,fleqn]{exam}
\usepackage{amssymb}
\usepackage[intlimits]{amsmath}
\usepackage{epsfig}
\usepackage{upgreek}
\usepackage[super]{nth}
\usepackage[colorlinks=true,linkcolor=black,anchorcolor=black,citecolor=black,filecolor=black,menucolor=black,runcolor=black,urlcolor=black]{hyperref}
\usepackage[letterpaper, margin=0.75in]{geometry}
\addpoints
\boxedpoints
\pointsinmargin
\pointname{pts}
\usepackage{tikz}
\usepackage{tkz-euclide}
\usetikzlibrary{shapes.geometric}
\usetikzlibrary{calc}
\usepackage[final]{microtype}
\frenchspacing
\usepackage[american]{babel}
\usepackage[T1]{fontenc}
\usepackage[]{fourier}

\usepackage{isomath}
\usepackage{upgreek,amsmath}
\usepackage{graphicx}

\newcommand{\dotprod}{\, {\scriptzcriptztyle\stackrel{\bullet}{{}}}\,}

\newcommand{\reals}{\mathbf{R}}
\newcommand{\lub}{\mathrm{lub}} 
\newcommand{\glb}{\mathrm{glb}} 
\newcommand{\complex}{\mathbf{C}}
\newcommand{\dom}{\mbox{dom}}
\newcommand{\range}{\mbox{range}}
\newcommand{\cover}{{\mathcal C}}
\newcommand{\integers}{\mathbf{Z}}
\newcommand{\vi}{\, \mathbf{i}}
\newcommand{\vj}{\, \mathbf{j}}
\newcommand{\vk}{\, \mathbf{k}}
\newcommand{\bi}{\, \mathbf{i}}
\newcommand{\bj}{\, \mathbf{j}}
\newcommand{\bk}{\, \mathbf{k}}
\DeclareMathOperator{\Arg}{\mathrm{Arg}}
\DeclareMathOperator{\Ln}{\mathrm{Ln}}
\newcommand{\imag}{\, \mathrm{i}}
\newcommand{\erf}{\mathrm{erf}}
\newcommand{\e}{\mathrm{e}}

\usepackage{graphicx}
\usepackage{color}
%\shadedsolutions
%\definecolor{SolutionColor}{rgb}{1,0.72,0.46} %{0.8,0.9,1}
\newcommand\AM{\textsc{am}}
\newcommand\PM{\textsc{pm}}


     
\newcommand{\quiz}{24}
\newcommand{\term}{Fall}
\newcommand{\due}{Tuesday 28 November 13:20}
\newcommand{\class}{MATH 202, Fall \the\year}
\begin{document}
\large
\noindent\makebox[3.0truein][l]{\textbf{\class}}
\textbf{Name:} \hrulefill \\
\noindent \makebox[3.0truein][l]{\textbf{In class work  \quiz}}
\textbf{Row and Seat}:\hrulefill\\



\noindent  In class work  \textbf{\quiz}  has questions \textbf{1} 
through  \textbf{\numquestions} \/ with a total of 
\textbf{\numpoints\/} points. Turn in your work at the end of class 
\emph{on paper}. This assignment is due at \emph{\due}.

\vspace{0.1in}



\noindent \emph{“Some people talk to animals. Not many listen though. 
That's the problem.”} \\  \phantom{xxx} \hfill {\sc   A. A. Milne}


\begin{questions} 
  
\question [1] The polar equation for a curve $\mathcal{C}$ is $r = 3 - 5 \sin(\vartheta)$.
Use Desmos to draw $\mathcal{C}$. As best you can, reproduce the curve
here.

\begin{solution}[2.5in]

\end{solution}

\question [1] From the graph, as best you can, find the cartesian
coordinates of each point on $\mathcal{C}$ that has a \emph{horizontal
tangent.}
\begin{solution}%[2.5in]

\end{solution}

\newpage
\question [1] Find the exact location of each horizontal
tangent. To do this, use the parametric representation $\mathcal{C} = \begin{cases} 
    x = \left(3 - 5 \sin(\vartheta)\right) \cos(\vartheta)\\
    y = \left(3 - 5 \sin(\vartheta)\right) \sin(\vartheta)
\end{cases}$. You will need to simultaneously solve $\displaystyle
\frac{\mathrm{d} y}{\mathrm{d} \vartheta} = 0 $ and $\displaystyle
\frac{\mathrm{d} x}{\mathrm{d} \vartheta} \neq 0$.
\end{questions}
    
\end{document}
