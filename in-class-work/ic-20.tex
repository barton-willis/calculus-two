\documentclass[12pt,fleqn]{exam}
\usepackage{amssymb}
\usepackage[intlimits]{amsmath}
\usepackage{epsfig}
\usepackage{upgreek}
\usepackage[super]{nth}
\usepackage[colorlinks=true,linkcolor=black,anchorcolor=black,citecolor=black,filecolor=black,menucolor=black,runcolor=black,urlcolor=black]{hyperref}
\usepackage[letterpaper, margin=0.75in]{geometry}
\addpoints
\boxedpoints
\pointsinmargin
\pointname{pts}
\usepackage{tikz}
\usepackage{tkz-euclide}
\usetikzlibrary{shapes.geometric}
\usetikzlibrary{calc}
\usepackage[final]{microtype}
\frenchspacing
\usepackage[american]{babel}
\usepackage[T1]{fontenc}
\usepackage[]{fourier}
\usepackage{isomath}
\usepackage{upgreek,amsmath}
\usepackage{graphicx}

\newcommand{\dotprod}{\, {\scriptzcriptztyle\stackrel{\bullet}{{}}}\,}

\newcommand{\reals}{\mathbf{R}}
\newcommand{\lub}{\mathrm{lub}} 
\newcommand{\glb}{\mathrm{glb}} 
\newcommand{\complex}{\mathbf{C}}
\newcommand{\dom}{\mbox{dom}}
\newcommand{\range}{\mbox{range}}
\newcommand{\cover}{{\mathcal C}}
\newcommand{\integers}{\mathbf{Z}}
\newcommand{\vi}{\, \mathbf{i}}
\newcommand{\vj}{\, \mathbf{j}}
\newcommand{\vk}{\, \mathbf{k}}
\newcommand{\bi}{\, \mathbf{i}}
\newcommand{\bj}{\, \mathbf{j}}
\newcommand{\bk}{\, \mathbf{k}}
\DeclareMathOperator{\Arg}{\mathrm{Arg}}
\DeclareMathOperator{\Ln}{\mathrm{Ln}}
\newcommand{\imag}{\, \mathrm{i}}
\newcommand{\erf}{\mathrm{erf}}

\usepackage{graphicx}
\usepackage{color}
%\shadedsolutions
%\definecolor{SolutionColor}{rgb}{1,0.72,0.46} %{0.8,0.9,1}
\newcommand\AM{\textsc{am}}
\newcommand\PM{\textsc{pm}}


     
\newcommand{\quiz}{20}
\newcommand{\term}{Fall}
\newcommand{\due}{Tuesday 7 November 13:20}
\newcommand{\class}{MATH 202, Fall \the\year}
\begin{document}
\large
\noindent\makebox[3.0truein][l]{\textbf{\class}}
\textbf{Name:} \hrulefill \\
\noindent \makebox[3.0truein][l]{\textbf{In class work  \quiz}}
\textbf{Row and Seat}:\hrulefill\\



\noindent  In class work  \textbf{\quiz}  has questions \textbf{1} 
through  \textbf{\numquestions} \/ with a total of 
\textbf{\numpoints\/} points. Turn in your work at the end of class 
\emph{on paper}. This assignment is due \emph{\due}.

\vspace{0.1in}
\noindent \emph{“There’s no problem so awful that you can’t add some guilt to it and make it even worse.”} \\ \phantom{xxx} \hfill {\sc Calvin (Bill Watterson)}


\begin{questions} 

\question[2] Find the Taylor polynomial of order four centered at zero for the cosine function; that is find the polynomial
$\displaystyle
   P_4(x) = \sum_{k=0}^4 \frac{\cos^{(k)}(0)}{k!} x^k.
$
You'll need to find the numerical values of $\cos^{(0)}(0), \cos^{(1)}(0),  \cos^{(2)}(0),  \cos^{(3)}(0),$ and $ \cos^{(4)}(0)$.
\begin{solution}[3.5in]

\end{solution}

\question[2] Use Desmos to graph $y = \cos(x)$ and $P_4$, the Taylor polynomial of order four centered at zero for the cosine function.
Reproduce the graph here.  For $x \in (-2,2)$, use the graph to estimate the maximum of $| \cos(x) - P_4(x)|$.

\begin{solution}%[2.5in]

\end{solution}

\newpage

\question[2] Find the Taylor polynomial of order four centered at zero for the \emph{square} of the cosine function.  You could \emph{suffer}
through the calculation by finding the first four derivatives of $\cos(x)^2$ and evaluate them at zero. Or you could use the fact we learned
in class on Monday.  For $k \in \integers_{\geq 0}$ and infinitely differentiable functions $F$ and $G$, define
\begin{equation*}
   a_k = \frac{F^{(k)}(0)}{k!}, \quad 
   b_k = \frac{G^{(k)}(0)}{k!}.
\end{equation*}
Then for all $n \in \integers_{\geq 0}$, we have
\begin{equation*}
   \frac{ \left(F G\right)^{(n)}(0)}{n!} = \sum_{k=0}^n  a_k b_{n-k}.
\end{equation*}
You will want to use these formulae with $F = \cos$ and $G = \cos$.  You can find the numbers $a_o, a_1, a_2, a_3, a_4$ 
and $b_o, b_1, b_2, b_3, b_4$ from the 
first question. 




\end{questions}
\end{document}
