\documentclass[12pt,fleqn]{exam}
%\usepackage{pifont}
%\usepackage{dingbat,bbding}

\usepackage{amssymb}
\usepackage[intlimits]{amsmath}
\usepackage{epsfig}
\usepackage{upgreek}
\usepackage[super]{nth}
\usepackage[colorlinks=true,linkcolor=black,anchorcolor=black,citecolor=black,filecolor=black,menucolor=black,runcolor=black,urlcolor=black]{hyperref}
\usepackage[letterpaper, margin=0.75in]{geometry}
\addpoints
\boxedpoints
\pointsinmargin
\pointname{pts}
\usepackage{tikz}
\usepackage{tkz-euclide}
\usetikzlibrary{shapes.geometric}
\usetikzlibrary{calc}
\usepackage[final]{microtype}
\frenchspacing
\usepackage[american]{babel}
\usepackage[T1]{fontenc}
\usepackage[]{fourier}
\usepackage{isomath}
\usepackage{upgreek,amsmath}
\usepackage{amssymb}
\usepackage{graphicx}

\newcommand{\dotprod}{\, {\scriptzcriptztyle\stackrel{\bullet}{{}}}\,}

\newcommand{\reals}{\mathbf{R}}
\newcommand{\lub}{\mathrm{lub}} 
\newcommand{\glb}{\mathrm{glb}} 
\newcommand{\complex}{\mathbf{C}}
\newcommand{\dom}{\mbox{dom}}
\newcommand{\range}{\mbox{range}}
\newcommand{\cover}{{\mathcal C}}
\newcommand{\integers}{\mathbf{Z}}
\newcommand{\vi}{\, \mathbf{i}}
\newcommand{\vj}{\, \mathbf{j}}
\newcommand{\vk}{\, \mathbf{k}}
\newcommand{\bi}{\, \mathbf{i}}
\newcommand{\bj}{\, \mathbf{j}}
\newcommand{\bk}{\, \mathbf{k}}
\DeclareMathOperator{\Arg}{\mathrm{Arg}}
\DeclareMathOperator{\Ln}{\mathrm{Ln}}
\newcommand{\imag}{\, \mathrm{i}}

\usepackage{graphicx}
\usepackage{color}
%\shadedsolutions
%\definecolor{SolutionColor}{rgb}{1,0.72,0.46} %{0.8,0.9,1}
\newcommand\AM{\textsc{am}}
\newcommand\PM{\textsc{pm}}
     
\newcommand{\quiz}{8}
\newcommand{\term}{Fall}
\newcommand{\due}{Thursday 21 September 13:20}
\newcommand{\class}{MATH 202, Fall \the\year}
\begin{document}
\large
\vspace{0.1in}
\noindent\makebox[3.0truein][l]{\textbf{\class}}
\textbf{Name:} \hrulefill \\
\noindent \makebox[3.0truein][l]{\textbf{In class work week \quiz}}
\textbf{Row and Seat}:\hrulefill\\
\vspace{0.1in}


\noindent  In class work  \textbf{\quiz\/}  has questions \textbf{1} through  \textbf{\numquestions} \/ with a total of \textbf{\numpoints\/}  points.   
Turn in your work at the end of class  \emph{on paper}. This assignment is due \emph{\due}.

\vspace{0.1in}


\begin{questions} 

\question [2] Find the area of the region $\{(x,y) | 0 \leq y \leq \sin(x)^2 \mbox{ and } 0 \leq x \leq \uppi \}$.

\begin{solution}[2.5in]
\end{solution}

\question [2] Find the numerical value of $\int_0^\uppi \cos(x)^3 \, \mathrm{d} x$.
\begin{solution}[2.5in]
\end{solution}

\question [2] Find the numerical value of $\int_0^\uppi \cos(x)^3 \, \mathrm{d} x$.
\begin{solution}%[2.5in]
\end{solution}

\newpage 
\question [3] Use the identities
\begin{align*}
    \sin{(x)} \cos{(y)} &= \frac{\sin{\left( y+x\right) }-\sin{\left( y-x\right) }}{2}, \\
    \sin{(x)} \sin{(y)} &=  -\frac{\cos{\left( y+x\right) }-\cos{\left( y-x\right) }}{2}, \\
    \cos{(x)} \cos{(y)} &= \frac{\cos{\left( y+x\right) }+\cos{\left( y-x\right) }}{2}.    
\end{align*}
to find the values of each of the following definite integrals

\begin{parts}
\part [2] $\int_0^{2 \uppi} \sin(5 x) \cos(x) \, \mathrm{d} x.$
\begin{solution}[2.5in]
\end{solution}

\part [2] $\int_0^{2 \uppi} \cos(5 x) \cos(x) \, \mathrm{d} x.$
\begin{solution}[2.5in]
\end{solution}



\part [2] $\int_0^{2 \uppi} \cos(5 x)^2  \, \mathrm{d} x.$

\begin{solution}%[2.5in]
\end{solution}
\end{parts}
\end{questions}

\end{document}

