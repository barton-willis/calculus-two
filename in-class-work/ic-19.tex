\documentclass[12pt,fleqn]{exam}
\usepackage{amssymb}
\usepackage[intlimits]{amsmath}
\usepackage{epsfig}
\usepackage{upgreek}
\usepackage[super]{nth}
\usepackage[colorlinks=true,linkcolor=black,anchorcolor=black,citecolor=black,filecolor=black,menucolor=black,runcolor=black,urlcolor=black]{hyperref}
\usepackage[letterpaper, margin=0.75in]{geometry}
\addpoints
\boxedpoints
\pointsinmargin
\pointname{pts}
\usepackage{tikz}
\usepackage{tkz-euclide}
\usetikzlibrary{shapes.geometric}
\usetikzlibrary{calc}
\usepackage[final]{microtype}
\frenchspacing
\usepackage[american]{babel}
\usepackage[T1]{fontenc}
\usepackage[upright]{fourier}
\usepackage{isomath}
\usepackage{upgreek,amsmath}
\usepackage{graphicx}

\newcommand{\dotprod}{\, {\scriptzcriptztyle\stackrel{\bullet}{{}}}\,}

\newcommand{\reals}{\mathbf{R}}
\newcommand{\lub}{\mathrm{lub}} 
\newcommand{\glb}{\mathrm{glb}} 
\newcommand{\complex}{\mathbf{C}}
\newcommand{\dom}{\mbox{dom}}
\newcommand{\range}{\mbox{range}}
\newcommand{\cover}{{\mathcal C}}
\newcommand{\integers}{\mathbf{Z}}
\newcommand{\vi}{\, \mathbf{i}}
\newcommand{\vj}{\, \mathbf{j}}
\newcommand{\vk}{\, \mathbf{k}}
\newcommand{\bi}{\, \mathbf{i}}
\newcommand{\bj}{\, \mathbf{j}}
\newcommand{\bk}{\, \mathbf{k}}
\DeclareMathOperator{\Arg}{\mathrm{Arg}}
\DeclareMathOperator{\Ln}{\mathrm{Ln}}
\newcommand{\imag}{\, \mathrm{i}}
\newcommand{\erf}{\mathrm{erf}}

\usepackage{graphicx}
\usepackage{color}
%\shadedsolutions
%\definecolor{SolutionColor}{rgb}{1,0.72,0.46} %{0.8,0.9,1}
\newcommand\AM{\textsc{am}}
\newcommand\PM{\textsc{pm}}
     
\newcommand{\quiz}{18}
\newcommand{\term}{Fall}
\newcommand{\due}{Thursday 2 November 13:20}
\newcommand{\class}{MATH 202, Fall \the\year}
\begin{document}
%\large
\vspace{0.1in}
\noindent\makebox[3.0truein][l]{\textbf{\class}}
\textbf{Name:} \hrulefill \\
\noindent \makebox[3.0truein][l]{\textbf{In class work  \quiz}}
\textbf{Row and Seat}:\hrulefill\\

\large
\noindent \emph{“
“Life is full of surprises, but never when you need one.” } \hfill {\sc Calvin (Bill Watterson)}

\vspace{0.1in}

\noindent  In class work  \textbf{\quiz}  has questions \textbf{1} 
through  \textbf{\numquestions} \/ with a total of 
\textbf{\numpoints\/} points. Turn in your work at the end of class 
\emph{on paper}. This assignment is due \emph{\due}.


 

\begin{questions} 

\question [1] Use the fact that for all real $x$ the equation $\displaystyle \mathrm{e}^x = \sum_{k=0}^\infty \frac{1}{k!}  x^k$ is an identity 
 to find a power series representation for $\mathrm{e}^{-x^2} $.
\begin{solution}[2.5in]

\end{solution}


\question [1] Define a function $\erf$ by the definite integral $\erf(x) = \frac{2}{\sqrt{\uppi}} \large \int_0^x \mathrm{e}^{-t^2} \, \mathrm{d} t$.
Find a power series representation for $\erf$.  Find the radius of convergence of this power series.
\begin{solution}%[3.5in]

\end{solution}

\newpage

\question [1]  With a bit of faith and luck, maybe a good approximation to $\erf$ is the sum of its first 101 terms.
Use Desmos to graph this approximation for $-6 < x < 6$.   Actually, the function $\erf$ is known to Desmos. Plot
a graph of both the sum of the first 101 terms of the power series for $\erf$ along with the function $\erf$.
As best you can, reproduce the graphs here.


\begin{solution}[1.5in]

\end{solution}


\question [1] Based on the Desmos graph of $\erf$, what is your conjecture for the value of $\displaystyle \lim_{x \to \infty} \erf(x)$? 


\begin{solution}%[4.5in]

\end{solution}

\newpage

\question [1]  Find the numerical value of $\displaystyle \lim_{x \to \infty}  \frac{2}{\sqrt{\uppi}}  \sum_{k=0}^{100} 
\frac{(-1)^k}{ (2 k + 1) (k !)}   x^{2 k +1}$.  For ``large'' values of $x$, explain why 
$\frac{2}{\sqrt{\uppi}}  \sum_{k=0}^{N} \frac{(-1)^k}{ (2 k + 1) (k !)}   x^{2 k +1}$ is not a good approximation to $\erf$
no matter how large we make $N$.  

\textbf{Remember:} No matter the degree of a polynomial, its limit toward infinity is determined 
by the term of the polynomial with the highest power.  For example, provided $a_{1 000 000} \neq 0$, we have 
\begin{equation*}
  \lim_{x \to \infty}  \left( a_o + a_1 x + a_2 x^2 + \cdots + a_{1 000 000} x^{1 000 000} \right)  =  \lim_{x \to \infty} a_{1 000 000} x^{1 000 000} .
\end{equation*}

\begin{solution}[3.5in]

\end{solution}

\question  [1] Find a formula for the derivative of $\erf$; that is find a formula for $\erf^\prime$.  Make your formula as ``simple'' as you
can.



\end{questions}
\end{document}
