\documentclass[12pt,fleqn,answers]{exam}
\usepackage{amssymb}
\usepackage[intlimits]{amsmath}
\usepackage{epsfig}
\usepackage{upgreek}
\usepackage[super]{nth}
\usepackage[colorlinks=true,linkcolor=black,anchorcolor=black,citecolor=black,filecolor=black,menucolor=black,runcolor=black,urlcolor=black]{hyperref}
\usepackage[letterpaper, margin=0.75in]{geometry}
\addpoints
\boxedpoints
\pointsinmargin
\pointname{pts}
\usepackage{tikz}
\usepackage{tkz-euclide}
\usetikzlibrary{shapes.geometric}
\usetikzlibrary{calc}
\usepackage[final]{microtype}
\frenchspacing
\usepackage[american]{babel}
\usepackage[T1]{fontenc}
\usepackage[upright]{fourier}
\usepackage{isomath}
\usepackage{upgreek,amsmath}
\usepackage{graphicx}

\newcommand{\dotprod}{\, {\scriptzcriptztyle\stackrel{\bullet}{{}}}\,}

\newcommand{\reals}{\mathbf{R}}
\newcommand{\lub}{\mathrm{lub}} 
\newcommand{\glb}{\mathrm{glb}} 
\newcommand{\complex}{\mathbf{C}}
\newcommand{\dom}{\mbox{dom}}
\newcommand{\range}{\mbox{range}}
\newcommand{\cover}{{\mathcal C}}
\newcommand{\integers}{\mathbf{Z}}
\newcommand{\vi}{\, \mathbf{i}}
\newcommand{\vj}{\, \mathbf{j}}
\newcommand{\vk}{\, \mathbf{k}}
\newcommand{\bi}{\, \mathbf{i}}
\newcommand{\bj}{\, \mathbf{j}}
\newcommand{\bk}{\, \mathbf{k}}
\DeclareMathOperator{\Arg}{\mathrm{Arg}}
\DeclareMathOperator{\Ln}{\mathrm{Ln}}
\newcommand{\imag}{\, \mathrm{i}}

\usepackage{graphicx}
\usepackage{color}
%\shadedsolutions
%\definecolor{SolutionColor}{rgb}{1,0.72,0.46} %{0.8,0.9,1}
\newcommand\AM{\textsc{am}}
\newcommand\PM{\textsc{pm}}
     
\newcommand{\quiz}{18}
\newcommand{\term}{Fall}
\newcommand{\due}{Tuesday 31 October 13:20}
\newcommand{\class}{MATH 202, Fall \the\year}
\begin{document}
%\large
\vspace{0.1in}
\noindent\makebox[3.0truein][l]{\textbf{\class}}
\textbf{Name:} \hrulefill \\
\noindent \makebox[3.0truein][l]{\textbf{In class work  \quiz}}
\textbf{Row and Seat}:\hrulefill\\

\normalsize
\noindent \emph{“
“Life is full of surprises, but never when you need one.” } \hfill {\sc Calvin }

\vspace{0.1in}

\noindent  In class work  \textbf{\quiz}  has questions \textbf{1} 
through  \textbf{\numquestions} \/ with a total of 
\textbf{\numpoints\/} points. Turn in your work at the end of class 
\emph{on paper}. This assignment is due \emph{\due}.

\vspace{0.1in}
As handy as the factorial might be, sometimes we need the product of the odd integers; that is,
we need a product of the form $1 \times 3 \times 5 \times \cdots \times (2 n - 1)$.  We'll define
\begin{equation}
   n !! = \prod_{k=1}^n (2 k -1),
\end{equation}
where $\prod$ means to find the product of the numbers. For example
\begin{align*}
   2 !! &= \prod_{k=1}^2 (2 k -1) = 1 \times 3 = 3, \\
   3 !! &= \prod_{k=1}^3 (2 k -1) = 1 \times 3 \times 5 = 15, \\
   4 !! &= \prod_{k=1}^4 (2 k -1) = 1 \times 3 \times 5 \times 7= 105. \\
  \begin{align*}
  It's not too hard to prove that for all positive integers $n$ that
  \begin{equation}
   \frac{(n+1) !!}{n !!}  = 2 n + 1.
\end{equation}


\begin{questions} 

\question Use the ratio test to find the radius of convergence of the series $\sum_{k=0}^\infty  \frac{(k + 1)!!}{2^k k!} x^k$.


\end{questions}
\end{document}
