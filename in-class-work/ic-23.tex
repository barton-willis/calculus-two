\documentclass[12pt,fleqn]{exam}
\usepackage{amssymb}
\usepackage[intlimits]{amsmath}
\usepackage{epsfig}
\usepackage{upgreek}
\usepackage[super]{nth}
\usepackage[colorlinks=true,linkcolor=black,anchorcolor=black,citecolor=black,filecolor=black,menucolor=black,runcolor=black,urlcolor=black]{hyperref}
\usepackage[letterpaper, margin=0.75in]{geometry}
\addpoints
\boxedpoints
\pointsinmargin
\pointname{pts}
\usepackage{tikz}
\usepackage{tkz-euclide}
\usetikzlibrary{shapes.geometric}
\usetikzlibrary{calc}
\usepackage[final]{microtype}
\frenchspacing
\usepackage[american]{babel}
\usepackage[T1]{fontenc}
\usepackage[]{fourier}

\usepackage{isomath}
\usepackage{upgreek,amsmath}
\usepackage{graphicx}

\newcommand{\dotprod}{\, {\scriptzcriptztyle\stackrel{\bullet}{{}}}\,}

\newcommand{\reals}{\mathbf{R}}
\newcommand{\lub}{\mathrm{lub}} 
\newcommand{\glb}{\mathrm{glb}} 
\newcommand{\complex}{\mathbf{C}}
\newcommand{\dom}{\mbox{dom}}
\newcommand{\range}{\mbox{range}}
\newcommand{\cover}{{\mathcal C}}
\newcommand{\integers}{\mathbf{Z}}
\newcommand{\vi}{\, \mathbf{i}}
\newcommand{\vj}{\, \mathbf{j}}
\newcommand{\vk}{\, \mathbf{k}}
\newcommand{\bi}{\, \mathbf{i}}
\newcommand{\bj}{\, \mathbf{j}}
\newcommand{\bk}{\, \mathbf{k}}
\DeclareMathOperator{\Arg}{\mathrm{Arg}}
\DeclareMathOperator{\Ln}{\mathrm{Ln}}
\newcommand{\imag}{\, \mathrm{i}}
\newcommand{\erf}{\mathrm{erf}}
\newcommand{\e}{\mathrm{e}}

\usepackage{graphicx}
\usepackage{color}
%\shadedsolutions
%\definecolor{SolutionColor}{rgb}{1,0.72,0.46} %{0.8,0.9,1}
\newcommand\AM{\textsc{am}}
\newcommand\PM{\textsc{pm}}


     
\newcommand{\quiz}{23}
\newcommand{\term}{Fall}
\newcommand{\due}{Tuesday 21 November 13:20}
\newcommand{\class}{MATH 202, Fall \the\year}
\begin{document}
\large
\noindent\makebox[3.0truein][l]{\textbf{\class}}
\textbf{Name:} \hrulefill \\
\noindent \makebox[3.0truein][l]{\textbf{In class work  \quiz}}
\textbf{Row and Seat}:\hrulefill\\



\noindent  In class work  \textbf{\quiz}  has questions \textbf{1} 
through  \textbf{\numquestions} \/ with a total of 
\textbf{\numpoints\/} points. Turn in your work at the end of class 
\emph{on paper}. This assignment is due at \emph{\due}.

\vspace{0.1in}

\noindent \emph{“Piglet noticed that even though he had a very small heart, 
it could hold a rather large amount of gratitude.”}
 \phantom{xxx} \hfill {\sc   A. A. Milne}


\begin{questions} 
    
    \question Consider the parametrically defined curve $x =  \frac{15 t}{1+t^3},
    \quad y = \frac{15 t^2}{1+ t^3}$


    \begin{parts}

    \part [1] Use Desmos to draw this curve. Reproduce the curve as best you can 
    on here:

    \begin{solution}[2.5in]

    \end{solution}

    \part [1] Find all points on the curve that have a horizontal tangent line. 
    You'll need to solve $\displaystyle \frac{\mathrm{d} y} {\mathrm{d} t} = 0$
    and $\displaystyle \frac{\mathrm{d} x} {\mathrm{d} t} \neq  0$. 
   


    \begin{solution}%[3.5in]

    \end{solution}

    \newpage
    \part [1] Find all points on the curve that have a vertical tangent line. 
    You'll need to solve $\displaystyle \frac{\mathrm{d} x} {\mathrm{d} t} = 0$
    and $\displaystyle \frac{\mathrm{d} y} {\mathrm{d} t} \neq  0$. 
  

    \begin{solution}[3.5in]

    \end{solution}

\end{parts}
\end{questions}
    
\end{document}
