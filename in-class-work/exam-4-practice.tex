\documentclass[12pt,fleqn]{exam}
\usepackage{amssymb}
\usepackage[intlimits]{amsmath}
\usepackage{epsfig}
\usepackage{upgreek}
\usepackage[super]{nth}
\usepackage[colorlinks=true,linkcolor=black,anchorcolor=black,citecolor=black,filecolor=black,menucolor=black,runcolor=black,urlcolor=black]{hyperref}
\usepackage[letterpaper, margin=0.75in]{geometry}
\addpoints
\boxedpoints
\pointsinmargin
\pointname{pts}
\usepackage{tikz}
\usepackage{tkz-euclide}
\usetikzlibrary{shapes.geometric}
\usetikzlibrary{calc}
\usepackage[final]{microtype}
\usepackage[american]{babel}
\usepackage[T1]{fontenc}
\usepackage[]{fourier}

\usepackage{isomath}
\usepackage{upgreek,amsmath}
\usepackage{graphicx}

\newcommand{\dotprod}{\, {\scriptzcriptztyle\stackrel{\bullet}{{}}}\,}

\newcommand{\reals}{\mathbf{R}}
\newcommand{\lub}{\mathrm{lub}} 
\newcommand{\glb}{\mathrm{glb}} 
\newcommand{\complex}{\mathbf{C}}
\newcommand{\dom}{\mbox{dom}}
\newcommand{\range}{\mbox{range}}
\newcommand{\cover}{{\mathcal C}}
\newcommand{\integers}{\mathbf{Z}}
\newcommand{\vi}{\, \mathbf{i}}
\newcommand{\vj}{\, \mathbf{j}}
\newcommand{\vk}{\, \mathbf{k}}
\newcommand{\bi}{\, \mathbf{i}}
\newcommand{\bj}{\, \mathbf{j}}
\newcommand{\bk}{\, \mathbf{k}}
\DeclareMathOperator{\Arg}{\mathrm{Arg}}
\DeclareMathOperator{\Ln}{\mathrm{Ln}}
\newcommand{\imag}{\, \mathrm{i}}
\newcommand{\erf}{\mathrm{erf}}
\newcommand{\e}{\mathrm{e}}

\usepackage{graphicx}
\usepackage{color}
%\shadedsolutions
%\definecolor{SolutionColor}{rgb}{1,0.72,0.46} %{0.8,0.9,1}
\newcommand\AM{\textsc{am}}
\newcommand\PM{\textsc{pm}}


     
\newcommand{\quiz}{25}
\newcommand{\term}{Fall}
\newcommand{\due}{Thursday 30 November at 13:20}
\newcommand{\class}{MATH 202, Fall \the\year}
\begin{document}
\large
\noindent\makebox[3.0truein][l]{\textbf{\class}}
\textbf{Name:} \hrulefill \\
\noindent \makebox[3.0truein][l]{\textbf{Exam 4 Practice}}
\textbf{Row and Seat}:\hrulefill\\





\noindent \emph{“If people sat outside and looked at 
the stars each night, I’ll bet they’d live a lot differently.”}\\
  $\phantom{xxx}$ \hfill {\sc Calvin (Bill Waterson)}


\begin{questions} 
    
  
\question For the parametricaly defined curve $\begin{cases}
      x = 3 t \\ y = 9 t^2 + t \end{cases} \, -\infty < t < \infty$,
      eliminate the parameter $t$. Sketch the resulting curve in
      the $xy$ cartesian coordinate system.

\begin{solution}[2.5in] Solving $  x = 3 t$ for $t$ yields $t = x/3$.
  Substituting this into  $y = 9 t^2 + t$ gives $y = x^2 + x / 3$.
  For a graph, try Desmos.
\end{solution}

\question For the parametricaly defined curve $\begin{cases}
    x = 3 \cos(t) \\ y = 9 \sin(t) \end{cases} \, 0 \leq  t \leq 2 \uppi$,
    eliminate the parameter $t$. Sketch the resulting curve in
    the $xy$ cartesian coordinate system.

\begin{solution}%[2.5in]
  $(x/3)^2 + (y/9)^2 = 1$.  For a graph, try Desmos.
\end{solution}

\newpage


\question For the curve $x^3 + y^3 - 3 x y = 48$, find the 
numerical value of $\displaystyle 
\left . \frac{\mathrm{d} y}{\mathrm{d}x} \right |_{x=2,y=4}$
\begin{solution}[3.5in]
  $\left . \frac{\mathrm{d} y}{\mathrm{d}x} \right |_{x=2,y=4} = 0$.
\end{solution}

\question For the parametrically defined curve 
$\displaystyle \begin{cases} x = -\sqrt{1+t} \\ y = \sqrt{3t} 
\end{cases} \, 0 \leq t < \infty$, find the numerical values 
of $\left . \frac{\mathrm{d} y}{\mathrm{d}x} \right |_{t=3}$
and $\left . \frac{\mathrm{d}^2 y}{\mathrm{d} x^2} \right |_{t=3}$.

\begin{solution}%[3.5in]
  See your class notes.
\end{solution}

\newpage 
\question Represent the arc length of the polar curve $r = a (1-\sin(\theta))$
as a definite integral. Here $a$ is a positive real number.

\begin{solution}[3.5in]
  $$
  \sqrt{2}\, \left| a\right|  \int_0^{2 \pi}   \sqrt{1-\sin{(\theta)}}
  \, \mathrm{d} \theta.
  $$
\end{solution}

\question Find all points on the polar curve $r = 3 - 2 \sin(\theta)$
that have a horizontal tangent line.

\begin{solution}%[3.5in]
 $\theta= \frac{\pi}{2}, \theta= \frac{\pi}{2},
  \theta = \operatorname{arcsin}\left( \frac{3}{4}\right) ,
  \theta = \pi - \operatorname{arcsin}\left( \frac{3}{4}\right)$
\end{solution}

\newpage 
\question Find area of the region bounded by the  polar curve 
$r = 3 - 2 \sin(\theta)$

\begin{solution}[3.5in]
 $\frac{11}{2} \pi.$
\end{solution}

\question Find the area bounded by the polar curve $r = \cos(4 \theta)$
\begin{solution}[3.5in]
  $\frac{\pi}{2}$
\end{solution}




\end{questions}

    
\end{document}
