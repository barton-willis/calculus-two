\documentclass[12pt,fleqn,answers]{exam}
%\usepackage{pifont}
%\usepackage{dingbat,bbding}

\usepackage{amssymb}
\usepackage[intlimits]{amsmath}
\usepackage{epsfig}
\usepackage{upgreek}
\usepackage[super]{nth}
\usepackage[colorlinks=true,linkcolor=black,anchorcolor=black,citecolor=black,filecolor=black,menucolor=black,runcolor=black,urlcolor=black]{hyperref}
\usepackage[letterpaper, margin=0.75in]{geometry}
\addpoints
\boxedpoints
\pointsinmargin
\pointname{pts}
\usepackage{tikz}
\usepackage{tkz-euclide}
\usetikzlibrary{shapes.geometric}
\usetikzlibrary{calc}
\usepackage[final]{microtype}
\frenchspacing
\usepackage[american]{babel}
\usepackage[T1]{fontenc}
\usepackage[]{fourier}
\usepackage{isomath}
\usepackage{upgreek,amsmath}
\usepackage{amssymb}
\usepackage{graphicx}

\newcommand{\dotprod}{\, {\scriptzcriptztyle\stackrel{\bullet}{{}}}\,}

\newcommand{\reals}{\mathbf{R}}
\newcommand{\lub}{\mathrm{lub}} 
\newcommand{\glb}{\mathrm{glb}} 
\newcommand{\complex}{\mathbf{C}}
\newcommand{\dom}{\mbox{dom}}
\newcommand{\range}{\mbox{range}}
\newcommand{\cover}{{\mathcal C}}
\newcommand{\integers}{\mathbf{Z}}
\newcommand{\vi}{\, \mathbf{i}}
\newcommand{\vj}{\, \mathbf{j}}
\newcommand{\vk}{\, \mathbf{k}}
\newcommand{\bi}{\, \mathbf{i}}
\newcommand{\bj}{\, \mathbf{j}}
\newcommand{\bk}{\, \mathbf{k}}
\DeclareMathOperator{\Arg}{\mathrm{Arg}}
\DeclareMathOperator{\Ln}{\mathrm{Ln}}
\newcommand{\imag}{\, \mathrm{i}}

\usepackage{graphicx}
\usepackage{color}
%\shadedsolutions
%\definecolor{SolutionColor}{rgb}{1,0.72,0.46} %{0.8,0.9,1}
\newcommand\AM{\textsc{am}}
\newcommand\PM{\textsc{pm}}
     
\newcommand{\quiz}{5}
\newcommand{\term}{Fall}
\newcommand{\due}{Thursday 7 September 13:20}
\newcommand{\class}{MATH 202, Fall \the\year}
\begin{document}
\large
\vspace{0.1in}
\noindent\makebox[3.0truein][l]{\textbf{\class}}
\textbf{Name:} \hrulefill \\
\noindent \makebox[3.0truein][l]{\textbf{In class work \quiz}}
\textbf{Row and Seat}:\hrulefill\\
\vspace{0.1in}


\noindent  In class work  \textbf{\quiz\/}  has questions \textbf{1} through  \textbf{\numquestions} \/ with a total of \textbf{\numpoints\/}  points.   
Turn in your work at the end of class  \emph{on paper}. This assignment is due \emph{\due}.

\vspace{0.1in}


\begin{questions} 

\question Find a formula for $F^\prime$ given

\begin{parts}

    \part [2] $F(x) =  \ln(x^2 + 1)$
    \begin{solution}[2.5in]  The chain rule when the outer function is the natural logarithm is
    \begin{equation*}
       \left(\ln(f(x)) \right)^\prime  = \frac{f^\prime(x)}{f(x)}.
     \end{equation*}
    Using this, we have
    \begin{align*}
       F^\prime(x)  &= \frac{(x^2 + 1)^\prime}{x^2 + 1}, && \mbox{(chain rule)} \\
                           &= \frac{2x }{x^2 + 1}. && \mbox{(polynomial derivative)} 
     \end{align*}
        
    \end{solution}

    \part [2] $F(x) = x \ln(x^2 + 1)$
    \begin{solution}[2.5in]
    For our first step, we need to use the product rule:
    \begin{align*}
       F^\prime(x)  &= \left(x\right)^\prime \ln(x^2 +1) + x   \left( \ln(x^2 +1) \right)^\prime,   
                                             && \mbox{(product rule)} \\
                         &=  \ln(x^2 +1) + x \frac{2x }{x^2 + 1},  &&\mbox{(polynomial derivative and part `a')} \\
                            &=  \ln(x^2 +1) + \frac{2x^2 }{x^2 + 1}.  &&\mbox{(generally viewed as simplification)}
\end{align*}
    \end{solution}

    \part [2] $F(x) = \frac{\ln(1+x) - \ln(1-x)}{2}$
    \begin{solution}
    For our first step, let's not use the quotient rule; instead  let's use outativity
    \begin{align*}
     F^\prime(x)  &= \frac{1}{2} \left(\ln(1+x) - \ln(1-x) \right)^\prime,  &&\mbox{(outative rule)}\\
                        &= \frac{1}{2} \left( \frac{1}{1+x} + \frac{1}{1-x} \right),  &&\mbox{(chain rule)} \\
                        &= \frac{1}{1-x^2}. &&\mbox{(generally viewed as simplification)} \\
    \end{align*}
        
    \end{solution}

\end{parts}
%\newpage

\question Find the numerical values of the definite integrals

\begin{parts}

    \part [2] $\int_{0}^1 \frac{x}{1+x^2} \, \mathrm{d} x$
    \begin{solution}[2.5in] The integrand doesn't match a standard integrands, so let's try a substitution.  
    Since we are substituting into a definite integral, there are \textbf{four} ingredients:
    \begin{enumerate}
        \item $z = 1 + x^2$
        \item $\mathrm{d} z = 2 x \mathrm{d} x$; alternatively $x \mathrm{d} x = \frac{1}{2} \mathrm{d} z$ 
        \item $x=0 \implies z = 1+0^2 = 1$
        \item $x=1 \implies z = 1+1^2 = 2$
    \end{enumerate}
    With that, we have
    \begin{align*}
    \int_{0}^1 \frac{x}{1+x^2} \, \mathrm{d} x &= \int_{1}^2 \frac{1}{2} \frac{1}{z} \, \mathrm{d} z, \\
               &= \left. \frac{1}{2} \ln(z) \right |_{z=1}^{z=2}, \\
               &= \frac{1}{2} \left( \ln(2) - \ln(1) \right) ,\\
                &= \frac{1}{2} \ln(2). \\
    \end{align*}
    Another answer is $\ln(\sqrt{2})$, but  one logarithm and one square root is less simple than is 
    one logarithm and one divide.   It's OK, to use $\int \frac{1}{z} \, \mathrm{d} z = \ln(|z|)$, but
    I recognized that the definite integral is over a positive interval, so forget about the absolute value.
    \end{solution}

    \part [2] $\int_{-4}^{-2} \frac{1}{x + 10} \, \mathrm{d} x$
    \begin{solution}[2.5in] The integrand doesn't match a standard integrands, so let's try a substitution.  
    Since we are substituting into a definite integral, there are \textbf{four} ingredients:
    \begin{enumerate}
        \item $z = x + 10$
        \item $\mathrm{d} z = \mathrm{d} x$
        \item $x=-4 \implies z = -4+10 = 6$
        \item $x=-2 \implies z = -2 + 10 = 8$
    \end{enumerate}
    So
    \begin{equation*}
    \int_{-4}^{-2} \frac{1}{x + 10} \, \mathrm{d} x = 
     \int_6^8 \frac{1}{z} \, \mathrm{d} z = \ln(8) - \ln(6) = \ln\left(\frac{4}{3}\right).
    \end{equation*}
    Since $ \ln(8) - \ln(6)$ involves two logarithms and $\ln\left(\frac{4}{3}\right)$ involves only
    one, generally,  $\ln\left(\frac{4}{3}\right)$ is a simplification of $\ln(8) - \ln(6)$
        
    \end{solution}

\end{parts}
\end{questions}

\end{document}

