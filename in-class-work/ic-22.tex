\documentclass[fleqn,answers,12pt]{exam}
\usepackage{pifont,enumerate,url}
\usepackage{dingbat}
\usepackage{amsmath}
\usepackage{amssymb}
\usepackage{fleqn}
\usepackage{epsfig,upgreek}
\usepackage{color}
\usepackage{nicefrac}
\usepackage{pdfpages}
\newcommand{\res}{\mathrm{res}}
\newenvironment{handlist}{
  \begin{enumerate}[\leftthumbsup]
    \addtolength{\itemsep}{-1.0\itemsep}}
  {\end{enumerate}}

\usepackage[letterpaper, margin=0.75in]{geometry}

% Let's put a proof inside a shaded box
\usepackage{amsthm,xcolor}
\usepackage{framed}

\colorlet{shadecolor}{lightgray!15}
\newenvironment{myproof}
  {\begin{shaded}\begin{proof}}
  {\end{proof}\end{shaded}}

\newtheorem{prop}{Proposition}

\usepackage{graphicx}
\usepackage{color}
\usepackage{tcolorbox}

\addpoints
\boxedpoints
\pointsinmargin
\pointname{pts}

\newcommand{\dotprod}{\, {\scriptzcriptztyle \stackrel{\bullet}{{}}}\,}
\newcommand{\complex}{\mathbf{C}}
\newcommand{\integers}{\mathbf{Z}}
\newcommand{\nat}{\mathbf{N}}
\newcommand{\imag}{\mathrm{i}}
\newcommand{\range}{\mathrm{range}}
\newcommand{\Res}{\mathrm{Res}}
\newcommand{\euler}{\mathrm{e}}
\newcommand{\lub}{\mathrm{lub}}
\newcommand{\ub}{\mathrm{ub}}
\newcommand{\true}{\mathrm{true}}
\newcommand{\erf}{\mathrm{erf}}
\usepackage{fourier}
%\shadedsolutions
\definecolor{SolutionColor}{rgb}{1,0.95,1}
%\definecolor{SolutionColor}{rgb}{1,1,0.7}
\addpoints
\boxedpoints
\pointsinmargin
\pointname{pts}
\newcommand{\reals}{\mathbf{R}}

\begin{document}

\large
\vspace{0.1in}
\noindent\makebox[3.0truein][l]{\textbf{MATH 202, Fall  \the\year}}
\textbf{Name:} \hrulefill \\
\noindent \makebox[3.0truein][l]{\textbf{In class work 22}}
\textbf{Row and Seat}:\hrulefill\\
\vspace{0.1in}

%\normalsize


\vspace{0.1in}



\begin{questions}
\question Use power series to find the numerical value of each limit. You might like to use the facts
\begin{align*}
\sin(x) &=  x\operatorname{-}\frac{{{x}^{3}}}{6}\operatorname{+}\frac{{{x}^{5}}}{120}\operatorname{+} \cdots, \\
\cos(x) &=  1\operatorname{-}\frac{{{x}^{2}}}{2}\operatorname{+}\frac{{{x}^{4}}}{24}\operatorname{-}\frac{{{x}^{6}}}{720}\operatorname{+} \cdots, \\
\ln(1+x) &= 
 x\operatorname{-}\frac{{{x}^{2}}}{2}\operatorname{+}\frac{{{x}^{3}}}{3}\operatorname{-}\frac{{{x}^{4}}}{4}\operatorname{+}\frac{{{x}^{5}}}{5}\operatorname{-}\frac{{{x}^{6}}}{6}\operatorname{+} \cdots \\
\sqrt{1+x} &= 1\operatorname{+}\frac{x}{2}\operatorname{-}\frac{{{x}^{2}}}{8}\operatorname{+}\frac{{{x}^{3}}}{16}\operatorname{-}\frac{5 {{x}^{4}}}{128}\operatorname{+}\frac{7 {{x}^{5}}}{256}\operatorname{-}\frac{21 {{x}^{6}}}{1024}\operatorname{+}\cdots \\
\arctan(x) &= x\operatorname{-}\frac{{{x}^{3}}}{3}\operatorname{+}\frac{{{x}^{5}}}{5}\operatorname{-}\frac{{{x}^{7}}}{7}\operatorname{+}\frac{{{x}^{9}}}{9}\operatorname{+} \cdots
\end{align*}

\begin{parts}

\part [1] $\displaystyle \lim_{x \to 0} \frac{\sin(x) - x + x^3/6}{x^5}$

\begin{solution}[2.5in]
\begin{equation*}
\lim_{x \to 0} \frac{\sin(x) - x + x^3/6}{x^5} = \lim_{x \to 0} \frac{ x^5/5! + \cdots}{x^5} = \frac{1}{120}.
\end{equation*}
\end{solution}

\part [1] $\displaystyle \lim_{x \to 0} \frac{ \sin(3 x^2)}{1 - \cos(2 x)}$

\begin{solution}%[2.5in]
\begin{equation*}
\lim_{x \to 0} \frac{ \sin(3 x^2)}{1 - \cos(2 x)} = \lim_{x \to 0} = 
\lim_{x \to 0}  \frac{3 x^2 + \cdots}{(2 x)^2 + \cdots} = \frac{3}{4}. 
\end{equation*}
\end{solution}
\end{parts}
\newpage

\question Use the \emph{ratio} test to determine the radius of convergence of each power series.


\begin{parts}

\part[1] $\displaystyle \sum_{k=0}^\infty \frac{(k!)^2}{ (2 k )!} x^k$

\begin{solution}[3.0in]
\begin{equation*}
\lim_{k \to \infty} \left | \frac{((k+1)!)^2 x^{k+1}}{ (2k + 2)! x^k} \right| =
\lim_{k \to \infty} \frac{k+1}{ 4 k + 2} |x| = \frac{|x|}{4}
\end{equation*}
The radius of convergence is 4.
\end{solution}

\part[1] $\displaystyle  \sum_{k=0}^\infty k ! x^k$

\begin{solution}[3.0in]

\begin{equation*}
\lim_{k \to \infty} \left |  \frac{(k+1)! x}{k!}   \right| =
\lim_{k \to \infty} (k+1) |x| = \begin{cases} \infty & x \neq 0 \\ 0 & x = 0 \end{cases}
\end{equation*}
The radius of convergence is zero.
\end{solution}





\end{parts}





\question [1] Find the numerical value of  \(\displaystyle \sum_{k=3}^\infty \left(\frac{1}{10} \right)^k \).
\begin{solution}%[2.5in]
\[
 \sum_{k=3}^\infty \left(\frac{1}{10} \right)^k = \sum_{k=3}^\infty \left(\frac{1}{10} \right)^{k + 3}
  = \frac{1}{10^3} \sum_{k=0}^\infty \left(\frac{1}{10} \right)^{k} = \frac{1}{10^3} \frac{1}{1- \frac{1}{10}} = 
  \frac{1}{900}
\]
\end{solution}
\newpage


\question The de Jonqui\'ere function \(\operatorname{Li}_4\) can be defined by 
\(\displaystyle
  \operatorname{Li}_4(x) = \sum_{k=1}^\infty \frac{x^k}{k^4}
\) and $\mathrm{dom}(\operatorname{Li}_4) = (-1,1)$

\begin{parts}

\part [1] Find the \emph{numerical value} of \(\operatorname{Li}_4(0)\).

\begin{solution}%[3.0in]
\[
 \operatorname{Li}_4(0) = \sum_{k=1}^\infty \frac{0^k}{k^4} = 0.
\]
\end{solution}

\newpage

\part [1] Find the \emph{numerical value} of \(\operatorname{Li}_4^\prime(0)\).
\begin{solution}[3.0in]
We have $\operatorname{Li}_4 (x) = x + \frac{x^2}{2^4} + \cdots$, so $\operatorname{Li}_4^\prime(0) =1$.
\end{solution}

\part [1] Find the \emph{numerical value} of \(\operatorname{Li}_4^{\prime \prime}(0)\).
\begin{solution}[3.0in]
We have $\operatorname{Li}_4 (x) = x + \frac{x^2}{2^4} + \cdots$, so $\operatorname{Li}_4^{\prime \prime}(0) =\frac{1}{8}$.
\end{solution}
\end{parts}


%\newpage
\question Determine convergence or divergence of each series.  Fully
justify your work quoting theorems from our class.
\begin{parts}
\part [1] \(\displaystyle \sum_{k=1}^\infty \frac{1}{8 k + 2} \)


\begin{solution}[2.5in] Since $\int_1^\infty \frac{1}{8 x + 2} \, \mathrm{d} x$ divereges, 
the series $\displaystyle \sum_{k=1}^\infty \frac{1}{8 k + 2}$ diverges.
\end{solution}

\part[1]  \(\displaystyle \sum_{k=0}^\infty (2 k + 1) (-1)^k  \)


\begin{solution}%[2.5in] 
The sequence \(k \mapsto (-1)^k (2 k + 1) \) does not converge to
zero; therefore the series \(\sum_{k=0}^\infty (2 k + 1) (-1)^k \) diverges.
The alternating series test, the integral test, and the ratio test
either do not apply or they give no information.
\end{solution}

\newpage

\part[1]   \(\displaystyle \sum_{k=0}^\infty \frac{(-1)^k}{2^k + 3^k} \)

\begin{solution}[2.5in]  This is a convergent alternating series.  We need to 
check three things.
\begin{itemize}

\item[\ding{52}] Is \(k \to \frac{1}{2^k + 3^k}\) a positive sequence? {\bf Yes},
\item [\ding{52}] Is \(k \to \frac{1}{2^k + 3^k}\) a decreasing sequence? 
{\bf Yes}

\item [\ding{52}] does  \(k \to \frac{1}{2^k + 3^k}\) converge to zero? {\bf Yes}
\end{itemize}
\end{solution}





\end{parts}

\question For all real numbers $x$, we have
$\displaystyle
   \cos(x) = \sum_{k=0}^\infty \frac{(-1)^k}{(2k)!} x^{2 k}.
$
Define a wild and crazy function ${Q}$ as  ${Q(x)} = \int_0^x \cos \left(t^4 \right)  \, \mathrm{d} t.$
\begin{parts}

\part [1] Find a power series centered at zero for ${Q}$.

\begin{solution}%[3.5in]
\begin{equation*}
Q(x) =  \int_0^x \sum_{k=0}^\infty \frac{(-1)^k}{(2 k)! } t^{2 k}  \, \mathrm{d} t
 = \sum_{k=0}^\infty \frac{(-1)^k}{(2 k)!  (2k) } x^{2 k + 1}
\end{equation*}
\end{solution}

\newpage

\part [1] Find the radius of convergence for the power series you found in the previous question.

\begin{solution}[3.5in] Since the radius of convergence of  $\sum_{k=0}^\infty \frac{(-1)^k}{(2k)!} x^{2 k}$
is infinty, so is the radius of convergence for the PS for $Q$.


\end{solution}

\end{parts}


\question [1] For all $x \in (-1,1)$, we have $\displaystyle \sqrt{1+x} = \sum_{k=0}^\infty \binom{\frac{1}{2}} {k} x^k$. Find the numerical value of
\begin{equation*} 
   \sum_{k = 0}^{42} \binom{\frac{1}{2}} {k} \binom{\frac{1}{2}} {42-k}.
\end{equation*}
   
   \begin{solution}  Let's define $a_k = \binom{\frac{1}{2}} {k}$. We have
   $\sqrt{1+x} \sqrt{1+x} = 1 + x = \sum_{k=0}^\infty c_k x^k$, where $c_k = \sum_{\ell = 0}^k a_\ell a_{k-\ell}$.
   So $\sum_{k = 0}^{42} \binom{\frac{1}{2}} {k} \binom{\frac{1}{2}} {42-k} = c_{42}$.  But $1 + x = 1 + x + 0 x^2 + 0 x^3 + \cdots$, so $c_{42} = 0$.
   
   \end{solution}
  
\end{questions}
\end{document}

