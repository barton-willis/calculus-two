\documentclass[12pt,fleqn,answers]{exam}
\usepackage{amssymb}
\usepackage[intlimits]{amsmath}
\usepackage{epsfig}
\usepackage{upgreek}
\usepackage[super]{nth}
\usepackage[colorlinks=true,linkcolor=black,anchorcolor=black,citecolor=black,filecolor=black,menucolor=black,runcolor=black,urlcolor=black]{hyperref}
\usepackage[letterpaper, margin=0.75in]{geometry}
\addpoints
\boxedpoints
\pointsinmargin
\pointname{pts}
\usepackage{tikz}
\usepackage{tkz-euclide}
\usetikzlibrary{shapes.geometric}
\usetikzlibrary{calc}
\usepackage[final]{microtype}
\frenchspacing
\usepackage[american]{babel}
\usepackage[T1]{fontenc}
\usepackage[]{fourier}
\usepackage{isomath}
\usepackage{upgreek,amsmath}
\usepackage{graphicx}

\newcommand{\dotprod}{\, {\scriptzcriptztyle\stackrel{\bullet}{{}}}\,}

\newcommand{\reals}{\mathbf{R}}
\newcommand{\lub}{\mathrm{lub}} 
\newcommand{\glb}{\mathrm{glb}} 
\newcommand{\complex}{\mathbf{C}}
\newcommand{\dom}{\mbox{dom}}
\newcommand{\range}{\mbox{range}}
\newcommand{\cover}{{\mathcal C}}
\newcommand{\integers}{\mathbf{Z}}
\newcommand{\vi}{\, \mathbf{i}}
\newcommand{\vj}{\, \mathbf{j}}
\newcommand{\vk}{\, \mathbf{k}}
\newcommand{\bi}{\, \mathbf{i}}
\newcommand{\bj}{\, \mathbf{j}}
\newcommand{\bk}{\, \mathbf{k}}
\DeclareMathOperator{\Arg}{\mathrm{Arg}}
\DeclareMathOperator{\Ln}{\mathrm{Ln}}
\newcommand{\imag}{\, \mathrm{i}}

\usepackage{graphicx}
\usepackage{color}
%\shadedsolutions
%\definecolor{SolutionColor}{rgb}{1,0.72,0.46} %{0.8,0.9,1}
\newcommand\AM{\textsc{am}}
\newcommand\PM{\textsc{pm}}
     
\newcommand{\quiz}{15}
\newcommand{\term}{Fall}
\newcommand{\due}{Thursday 19 October 13:20}
\newcommand{\class}{MATH 202, Fall \the\year}
\begin{document}
\large
\vspace{0.1in}
\noindent\makebox[3.0truein][l]{\textbf{\class}}
\textbf{Name:} \hrulefill \\
\noindent \makebox[3.0truein][l]{\textbf{In class work  \quiz}}
\textbf{Row and Seat}:\hrulefill\\

\noindent \emph{“The pencil is mightier than the pen.”} \hfill 
   \\  $\phantom{xxx}$ \hfill {\sc Robert M. Pirsig}


\noindent  In class work  \textbf{\quiz}  has questions \textbf{1} 
through  \textbf{\numquestions} \/ with a total of 
\textbf{6} points. Turn in your work at the end of class 
\emph{on paper}. This assignment is due \emph{\due}.

\vspace{0.1in}

\noindent \textbf{Warning:} For the most part, I've only given answers, not solutions.
This allows you to check your answers. Of course, for the exam, 
you must show all of your work.

\begin{questions} 
    


        \question Use \emph{integration by parts} to find an antiderivative of 
        each of the following:
    
        \begin{parts}
    
             \part   $\int x \mathrm{e}^{-x}  \, \mathrm{d} x$
    
             \begin{solution}[3.0in]
              \begin{equation*}
              \int x \mathrm{e}^{-x}  \, \mathrm{d} x = 
              -\left( x+1\right) \, {\mathrm{e}^{-x}}
            \end{equation*} 
             \end{solution}
    
             \part   $\int x^2 \mathrm{e}^{-x}   \, \mathrm{d} x$
    
             \begin{solution}%[3.0in]
                \begin{equation*}
                    \int x^2 \mathrm{e}^{-x}   \, \mathrm{d} x = 
                    -\left( {{x}^{2}}+2 x+2\right) \, {\mathrm{e}^{-x}}
                \end{equation*}
               
             \end{solution}
             
    
        \end{parts}
    
        \newpage 
        
        \question Define a region of the xy plane $Q$ by $Q = \{(x,y) |
            0 \leq y \leq  \mathrm{e}^{-x}  \mbox{ and } 0 \leq x \leq 5  \}$. \textbf{Hint:} For both parts 
            of this question, use an answer from
            Question 1. 
        \begin{parts}
      
            \part  Find $\mbox{Area}(Q)$
            \begin{solution}[3.0in]
                \begin{equation*}
                    \mbox{Area}(Q) = \int_0^5   \mathrm{e}^{-x} \, \mathrm{d} x 
                     = 1- \frac{1}{\mathrm{e}^5}
                \end{equation*}  
            \end{solution}
    
    
            \part  Find the x coordinate of the centroid of $Q$.
            \begin{solution}%[3.0in]
                \begin{equation*}
                \mbox{Area}(Q)  \, \overline{x} = \int_0^5  x \mathrm{e}^{-x} \, \mathrm{d} x
                   =  1 - \frac{6}{\mathrm{e}^5}.
                \end{equation*}
                So 
                \begin{equation*}
                   \overline{x} = \frac{1 - \frac{6}{\mathrm{e}^5}}{1- \frac{1}{\mathrm{e}^5}}.
                \end{equation*}
            \end{solution}
        \end{parts}
    
        \newpage
        \question Find a formula for each antiderivative.
        \begin{parts}
           \part   $
            \int \frac{x+9}{\left( x+4\right) \, \left( x+5\right) } \, \mathrm{d} x$ (Use partial
             fractions).
    
             \begin{solution}[3.0in]
                \begin{equation*}
                    \int \frac{x+9}{\left( x+4\right) \, \left( x+5\right) } \, \mathrm{d} x = 
                    5 \ln{\left( \left| x+4\right| \right) }-4 \ln{\left( \left| x+5\right| \right) }.
                \end{equation*}
        \end{solution}
    
             \part   $\int \frac{{{x}^{3}}}{\sqrt{1\operatorname{-}{{x}^{2}}}} \, 
             \mathrm{d}x$. (Use the substitution $x = \sin(\theta)$, where 
             $\theta \in (-\frac{\uppi}{2}, \frac{\uppi}{2})$.)
             \begin{solution}%[3.0in]
             It's OK to leave the expression in terms of a composition of a trigonometric function with an
             inverse trigonometric function.  But here is it is explicitly as a algebraic function: 
                \begin{equation*}
                    \int \frac{{{x}^{3}}}{\sqrt{1\operatorname{-}{{x}^{2}}}} \, 
             \mathrm{d}x = -\frac{{{x}^{2}}\, \sqrt{1-{{x}^{2}}}}{3}-\frac{2 \sqrt{1-{{x}^{2}}}}{3}.
                \end{equation*}
                And some alternative solutions too 
                \begin{align*}
                    \int \frac{{{x}^{3}}}{\sqrt{1\operatorname{-}{{x}^{2}}}} \, 
             \mathrm{d}x &= \frac{1}{3} (1-x^2)^{3/2} - (1-x^2)^{1/2}, \\
                         &= \frac{1}{3} \cos(\arcsin(x))^3 - \cos(\arcsin(x)).
                \end{align*}
                All three answers are OK. Please exercise your first 
                amendment rights and choose the solution you like the most.
             \end{solution}
        \end{parts} 
    \newpage
    
    \question Find the limit of each sequence $a$ whose formula is
    
    \begin{parts}
    
        \part  $a_n = \frac{(2n-1)(7 n + 1)}{n^2+1}$
    
        \begin{solution}[2.5in]
            \begin{equation*}
                \lim_{n \to \infty} a_n = 14.
            \end{equation*}
        \end{solution}
    
        \part   $a_n = n \ln \left(1 + \frac{\sqrt{2}}{n} \right)$
    
        \begin{solution}%[2.5in]
            \begin{equation*}
                \lim_{n \to \infty} a_n = \sqrt{2}.
            \end{equation*}
        \end{solution}
    
        \newpage
    
        \part   $a_n = \sqrt{n^2 + 46 n+ 1} - n$
    
        \begin{solution}[2.5in]
            \begin{equation*}
                \lim_{n \to \infty} a_n = 23.
            \end{equation*}
        \end{solution}
    
    \end{parts}
    \question   Give an example of a sequence $a$ such that
    $\displaystyle \lim_{k \to \infty} a_k = 0$ and $\displaystyle \lim_{n \to \infty} \sum_{k=1}^n a_k  = \infty$.
    
    \begin{solution}[1.50in]
    An example is $a_k = \frac{1}{k}$.  We have 
    $\displaystyle \lim_{k \to \infty} \frac{1}{k} = 0$ and 
        \begin{equation*}
              \lim_{n\to \infty}  \sum_{k=1}^n \frac{1}{k} = \infty.
        \end{equation*}
    \end{solution}
    \question   Give an example of a sequence $a$ such that
    $\displaystyle \lim_{k \to \infty} a_k = 0$ and $\displaystyle \lim_{n \to \infty} \sum_{k=1}^n a_k$ is a real 
    number.
    
    \begin{solution}%[1.5in]
    An example is $a_k = \frac{1}{k^2}$. Then 
        \begin{equation*}
            \lim_{n \to \infty} \sum_{k=1}^n \frac{1}{k^2}  \text{ converges }.
      \end{equation*}
    \end{solution}
    
    \newpage 
    \question   Show that the series $\displaystyle \sum_{k=1}^\infty \sqrt{k^2 + 46 k+ 1} - k$
    diverges. Justify your answer.
    \begin{solution}[2.50in]
        Since $\displaystyle \lim_{k \to \infty} \left(\sqrt{k^2 + 46 k+ 1} - k \right) \neq 0$, 
        the  series $\displaystyle \sum_{k=1}^\infty \sqrt{k^2 + 46 k+ 1} - k$ diverges.
    \end{solution}
    
    \question   Find the numerical value of the sum 
    $\sum_{k=0}^\infty 5 \left(\frac{2}{3} \right)^k$.
    
    \begin{solution}%[2.50in]
        $\sum_{k=0}^\infty 5 \left(\frac{2}{3} \right)^k = 15$
    \end{solution}
    
    \newpage 
    
    \question Find the numerical value for each improper integral.
    
    \begin{parts}
    
        \part   $\displaystyle \int_{-\infty}^\infty  \frac{1}{81+x^2} \, \mathrm{d} x$.
        \begin{solution}[3.50in]
            $\displaystyle \int_{-\infty}^\infty \frac{1}{81+x^2} \, \mathrm{d} x 
               = \frac{\uppi}{9}$
        \end{solution}
    
        \part   $\displaystyle \int_0^\infty \sin(x) \mathrm{e}^{-x} \, \mathrm{d} x$.
        \begin{solution}[3.50in]
            $\displaystyle \int_0^\infty \sin(x) \mathrm{e}^{-x} \, \mathrm{d} x = \frac{1}{2}$
        \end{solution}
    \end{parts}
    

\end{questions}
\end{document}
