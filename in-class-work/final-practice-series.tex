\documentclass[12pt,fleqn]{exam}
\usepackage{amssymb}
\usepackage[intlimits]{amsmath}
\usepackage{epsfig}
\usepackage{upgreek}
\usepackage[super]{nth}
\usepackage[colorlinks=true,linkcolor=black,anchorcolor=black,citecolor=black,filecolor=black,menucolor=black,runcolor=black,urlcolor=black]{hyperref}
\usepackage[letterpaper, margin=0.75in]{geometry}
\addpoints
\boxedpoints
\pointsinmargin
\pointname{pts}
\usepackage{tikz}
\usepackage{tkz-euclide}
\usetikzlibrary{shapes.geometric}
\usetikzlibrary{calc}
\usepackage[final]{microtype}
\usepackage[american]{babel}
\usepackage[T1]{fontenc}
\usepackage[]{fourier}

\usepackage{isomath}
\usepackage{upgreek,amsmath}
\usepackage{graphicx}

\newcommand{\dotprod}{\, {\scriptzcriptztyle\stackrel{\bullet}{{}}}\,}

\newcommand{\reals}{\mathbf{R}}
\newcommand{\lub}{\mathrm{lub}} 
\newcommand{\glb}{\mathrm{glb}} 
\newcommand{\complex}{\mathbf{C}}
\newcommand{\dom}{\mbox{dom}}
\newcommand{\range}{\mbox{range}}
\newcommand{\cover}{{\mathcal C}}
\newcommand{\integers}{\mathbf{Z}}
\newcommand{\vi}{\, \mathbf{i}}
\newcommand{\vj}{\, \mathbf{j}}
\newcommand{\vk}{\, \mathbf{k}}
\newcommand{\bi}{\, \mathbf{i}}
\newcommand{\bj}{\, \mathbf{j}}
\newcommand{\bk}{\, \mathbf{k}}
\DeclareMathOperator{\Arg}{\mathrm{Arg}}
\DeclareMathOperator{\Ln}{\mathrm{Ln}}
\newcommand{\imag}{\, \mathrm{i}}
\newcommand{\erf}{\mathrm{erf}}
\newcommand{\e}{\mathrm{e}}

\usepackage{graphicx}
\usepackage{color}
%\shadedsolutions
%\definecolor{SolutionColor}{rgb}{1,0.72,0.46} %{0.8,0.9,1}
\newcommand\AM{\textsc{am}}
\newcommand\PM{\textsc{pm}}


     
\newcommand{\quiz}{25}
\newcommand{\term}{Fall}
\newcommand{\due}{Thursday 30 November at 13:20}
\newcommand{\class}{MATH 202, Fall \the\year}
\begin{document}
\large
\noindent\makebox[3.0truein][l]{\textbf{\class}}
\textbf{Name:} \hrulefill \\
\noindent \makebox[3.0truein][l]{\textbf{Exam 4 Practice}}
\textbf{Row and Seat}:\hrulefill\\





\noindent \emph{“We seem to understand the value of oil, timber, minerals and housing, but not the 
value of unspoiled beauty, wildlife, solitude and spiritual renewal.”}\\
  $\phantom{xxx}$ \hfill {\sc Calvin (Bill Waterson)}


\begin{questions} 
    
 
\question Find the radius of convergence of each power series

\begin{parts}

    \part $\sum_{k=0}^\infty k! x^k$
    \begin{solution}[3.5in] For $x \neq 0$, the ratio test gives
        \begin{align*}
            \lim_{k \to \infty} \left| 
                  \frac {(k+1)! x^{k+1}}{ k! x^k} \right| 
                  &=  \lim_{k \to \infty} |k+1| |x|, \\
                  &= \infty.
        \end{align*}
    So the series  $\sum_{k=0}^\infty k! x^k$ does not converge
    for any nonzero real number $x$; so the radius of 
    convergence for $\sum_{k=0}^\infty k! x^k$ is zero.

    \end{solution} 
    
    \part $\sum_{k=0}^\infty \frac{5 k+1}{7 k + 2} (x-1)^k$
    \begin{solution}[3.5in] For $x \neq 1$, we have
        \begin{align*}
            \lim_{k \to \infty} \left|  
                  \frac{5 k + 6}{7 k + 9}
                  \frac{7 k + 2}{5 k + 1} |x| 
                     \right| 
                  &=  |x|
        \end{align*}
    So the series $\sum_{k=0}^\infty \frac{5 k+1}{7 k + 2} (x-1)^k$
    converges when $|x| < 1$. And that makes the radius of convergence
    1.

    \end{solution}

    \part $\sum_{k=0}^\infty \frac{(3k)!}{(k!)^3} \left(\frac{x}{5}\right)^k$
    \begin{solution}[3.5in] Let's do some simplifications first; we have
        \begin{align*}
              \frac{(3 k +3)!}{ (3k)! } &= \frac{ (3k+1)(3k+2)(3k+3)}, \\
              \frac{((k+1)!)^3 }{ (k!)^3 } &=  (k+1)^3.
        \end{align*}
        Now with these ingredients, we have 

    

    \end{solution}

\end{parts}

\question Find a power series representation for the function 
$G(x) = \int_0^x \exp(-t^4) \, \mathrm{d} t$.  Also, find the
radius of convergence of this power series.
\begin{solution}[3.5in]

\end{solution}

\question Determine if each series converges or diverges.

\begin{parts}

    \part $\sum_{k=0}^\infty \frac{1}{1+k^4}$
    \begin{solution}[3.5in]

    \end{solution}

    \part $\sum_{k=0}^\infty (-1)^k \frac{1}{1+k}$
    \begin{solution}[3.5in]

    \end{solution}


    \part $\sum_{k=0}^\infty \frac{\sin(k)}{1+k^2}$
    \begin{solution}[3.5in]

    \end{solution}



\end{parts}

\end{questions}
\end{document}




