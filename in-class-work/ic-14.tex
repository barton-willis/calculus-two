\documentclass[12pt,fleqn,answers]{exam}
\usepackage{amssymb}
\usepackage[intlimits]{amsmath}
\usepackage{epsfig}
\usepackage{upgreek}
\usepackage[super]{nth}
\usepackage[colorlinks=true,linkcolor=black,anchorcolor=black,citecolor=black,filecolor=black,menucolor=black,runcolor=black,urlcolor=black]{hyperref}
\usepackage[letterpaper, margin=0.75in]{geometry}
\addpoints
\boxedpoints
\pointsinmargin
\pointname{pts}
\usepackage{tikz}
\usepackage{tkz-euclide}
\usetikzlibrary{shapes.geometric}
\usetikzlibrary{calc}
\usepackage[final]{microtype}
\frenchspacing
\usepackage[american]{babel}
\usepackage[T1]{fontenc}
\usepackage[]{fourier}
\usepackage{isomath}
\usepackage{upgreek,amsmath}
\usepackage{graphicx}

\newcommand{\dotprod}{\, {\scriptzcriptztyle\stackrel{\bullet}{{}}}\,}

\newcommand{\reals}{\mathbf{R}}
\newcommand{\lub}{\mathrm{lub}} 
\newcommand{\glb}{\mathrm{glb}} 
\newcommand{\complex}{\mathbf{C}}
\newcommand{\dom}{\mbox{dom}}
\newcommand{\range}{\mbox{range}}
\newcommand{\cover}{{\mathcal C}}
\newcommand{\integers}{\mathbf{Z}}
\newcommand{\vi}{\, \mathbf{i}}
\newcommand{\vj}{\, \mathbf{j}}
\newcommand{\vk}{\, \mathbf{k}}
\newcommand{\bi}{\, \mathbf{i}}
\newcommand{\bj}{\, \mathbf{j}}
\newcommand{\bk}{\, \mathbf{k}}
\DeclareMathOperator{\Arg}{\mathrm{Arg}}
\DeclareMathOperator{\Ln}{\mathrm{Ln}}
\newcommand{\imag}{\, \mathrm{i}}

\usepackage{graphicx}
\usepackage{color}
%\shadedsolutions
%\definecolor{SolutionColor}{rgb}{1,0.72,0.46} %{0.8,0.9,1}
\newcommand\AM{\textsc{am}}
\newcommand\PM{\textsc{pm}}
     
\newcommand{\quiz}{14}
\newcommand{\term}{Fall}
\newcommand{\due}{Thursday 12 October 13:20}
\newcommand{\class}{MATH 202, Fall \the\year}
\begin{document}
\large
\vspace{0.1in}
\noindent\makebox[3.0truein][l]{\textbf{\class}}
\textbf{Name:} \hrulefill \\
\noindent \makebox[3.0truein][l]{\textbf{In class work  \quiz}}
\textbf{Row and Seat}:\hrulefill\\

\noindent \emph{“You are never dedicated to something you have complete confidence in.''} \hfill 
   \\  $\phantom{xxx}$ \hfill {\sc Robert M. Pirsig}


\noindent  In class work  \textbf{\quiz}  has questions \textbf{1} 
through  \textbf{\numquestions} \/ with a total of 
\textbf{\numpoints\/}  points.Turn in your work at the end of class 
\emph{on paper}. This assignment is due \emph{\due}.

\vspace{0.1in}


\begin{questions} 
    
   \question[2] When Morwenna graduates from UNK and starts her first job, she expects to earn a 
   starting annual salary of \$42,000. She plans to work for 42 years
   and she expects to earn a 3\% raise each year. Thus, in her 
   $\mathrm{n}^{\mathrm{th}}$  year 
   of work, her salary is $42,000 \times 1.03^{n-1}$. During 
   Morwenna's 42 years of labor, how much will she earn?

   \begin{solution}[2.5in] In this problem it's easy to get tangled up 
      with errors between 41 and 42, and between a sum index that starts 
      at zero or that starts at one. To navigate these problems, I suggest 
      expressing the lifetime amount Morwenna earns as simply as possible;
      for example\footnote{In 42 years of labor, Morwenna will earn 41 (not 42) raises. That 
      explains why the final term in the sum is $42,000 \times 1.03^{41}$
      and not $42,000 \times 1.03^{42}$}
      \begin{equation*}
         42,000 + 42,000 \times 1.03 + 42,000 \times 1.03^2 + \cdots 
         + 42,000 \times 1.03^{41}.
      \end{equation*}
   Expressed this way, it's most natural to make the lower sum index be 
   zero, not one, I think. And a bonus to using a lower sum index of zero 
   is that way, the sum explicitly matches the geometric sum identity in the 
   QRS. 
   

   In summation notation, the lifetime earnings for Morwenna is 
   \begin{equation*}
     \sum_{n=0}^{41} 42,000 \times 1.03^n.
   \end{equation*}
   To express this explicitly as a number, we need to use outativity
   along with the geometric sum identity; we have
   \begin{align*}
      \sum_{n=0}^{41} 42,000 \times 1.03^n &= 42,000 \sum_{n=0}^{41}  1.03^n, && \text{(outative property)} \\
                  &= 42,000 \frac{1 - 1.03^{42}}{1-1.03},  &&\text{(geometric sum identity)} \\
                  &= 3,444,974.25.  && \text{(round to the nearest penny)}
   \end{align*}
      \end{solution}

   \question Given a formula for a sequence $b$, find its limit.
   Show all of your work.

   \begin{parts}

    \part [2] $\displaystyle b_n = \sum_{k=0}^{n} \left(\frac{2}{3}\right)^k$.
    \begin{solution}%[2.5in]
      We don't have any rules for finding the limit of a sequence 
      whose formula is a sum. For us to have any chance of solving 
      this problem, we need to find an alternative formula for the
      sequence $b$ that doesn't involve a summation.

      \quad Fortunately, the formula for $b$ is a geometric sum. And 
      we have a nifty way to simplify to summation; we have
      \begin{equation*}
         \sum_{k=0}^{n} \left(\frac{2}{3}\right)^k =
            \frac{1 - \left(\frac{2}{3} \right)^{n+1}}{1 - \frac{2}{3}}
            = 3 \left (1 - \left(\frac{2}{3} \right)^{n+1} \right)
      \end{equation*}
   I know what you are thinking:  The QRS formula tells us that for for all positive integers $n$ and for all $x \neq 1$, we have $\sum_{k=0}^{n-1} x^k = \frac{1-x^n}{1-x}$, but
   we need $\sum_{k=0}^{n} x^k$. Are we toast?  No way.  The upper sum index in $\sum_{k=0}^{n-1} x^k = \frac{1-x^n}{1-x}$ can be
   replaced by any positive integer. But we have to remember that the mathematician standard: what we do to the left, we need to do to the
   right.  So we can replace every $n$ by $n+1$ in the identity $\sum_{k=0}^{n-1} x^k = \frac{1-x^n}{1-x}$ to determine that
   $\sum_{k=0}^{n} x^k = \frac{1-x^{n+1}}{1-x}$.
   
   \quad Finding the limit now is possible; we have
   \begin{equation}
      \lim_{n \to \infty} b_n =  \lim_{n \to \infty} 3 \left (1 - \left(\frac{2}{3} \right)^{n+1} \right) = 3.
  \end{equation}
  The base of the exponential term  $\left(\frac{2}{3} \right)^{n+1}$ is in the interval $(-1,1)$, so this 
  is a decaying exponential term.  And its limit is zero.
    
    \end{solution}

    \newpage 
    \part [2] $\displaystyle b_n = \sum_{k=0}^{n} \left(\frac{3}{2}\right)^k$.
    \begin{solution}%[2.5in]
    The strategy for this problem is much the same as before; we have
       \begin{equation}
      \lim_{n \to \infty} b_n =  \lim_{n \to \infty} \frac{1-  \left( \frac{3}{2}\right)^{n+1} }{1-\frac{3}{2}} = \infty.
  \end{equation}
    The base of the exponential term  $\left(\frac{3}{2} \right)^{n+1}$ is outside the interval $[-1,1]$, so this 
  is a growing  exponential term.  And its limit is infinity.
    
    \end{solution}






   \end{parts}
 
   %\newpage

   \question [2] The Newton–Raphson method\footnote{Raphson invented
   the method before Newton. If you didn't learn the Newton–Raphson method
   in Calculus I, I should tell the Office of Student Records to 
   expunge your MATH 115 credit.} is a way to find an 
   approximate solution to an equation $F(x)=0$. Specifically,
   the method starts with a guess for the solution, call it $a_1$,
   and then refines the guess with values $a_2, a_3, \dots$. 
   This sequence is called a \emph{Newton sequence}.
   If all goes well, the sequence $a$ converges to 
   solution to $F(x)=0$. Specifically for $F(x) = x^2 - 2$ and
   an initial guess of $1$, the Newton sequence  is defined
   recursively by 
   \begin{equation*}
    a_{n+1} = \begin{cases} 1  & n=0  \\
                            a_n - \frac{a_n^2 - 2}{2 a_n} & n > 0
    \end{cases}.
\end{equation*}
Assuming that the sequence $a$ converges to a positive number, find the numerical
value of $\displaystyle \lim_{n \to \infty} a_n$.  Use the 
fact that if $\displaystyle \lim_{n \to \infty} a_n  = L$,
then $\displaystyle \lim_{n \to \infty} a_{n+1} = L$.

\quad Many of you will have the urge to``simplify'' $a_n - \frac{a_n^2 - 2}{2 a_n}$
to $\frac{a_n^2 + 2}{2 a_n}$ or possibly to 
$\frac{a_n}{2} + \frac{1}{a_n}$. Doing so is an OK thing to do,
but I suggest doing a bit more `T' from GNAT\footnote{GNAT = Graphical,
Numerical, Algebraic, Think. I possibly invented the acronym, but 
the concept was invented by Deborah Hughes Hallett.} before you give into your urge to simplify.

\begin{solution}
Equal things have equal limits. So we have
\begin{equation*}
  \lim_{n \to \infty} a_{n+1}   = \lim_{n \to \infty} \left(a_n - \frac{a_n^2 - 2}{2 a_n}\right).
\end{equation*}
The usual rules of limits tells us that this equation is equivalent to 
\begin{equation*}
  L   = L  - \frac{L^2 - 2}{2 L}
\end{equation*}
So assuming the sequence $a$ converges to a positive number,  the sequence $a$ converges to $\sqrt{2}$.

Incidentally:  Every term in the sequence $a$ is a rational number; specifically, the first few terms are
\begin{equation*}
1,\frac{3}{2},\frac{17}{12},\frac{577}{408},\frac{665857}{470832},\frac{886731088897}{627013566048},\frac{1572584048032918633353217}{1111984844349868137938112}, \dots
\end{equation*}
The fact that sequence  whose range is in the set of rational number converges to an irrational number is a deep idea in mathematics. Also
rounded to 15 decimal digits, we have
\begin{equation*}
\frac{1572584048032918633353217}{1111984844349868137938112} \approx 1.414213562373095
\end{equation*}
And $1.414213562373095$ is very close to $\sqrt{2}$.  We don't have a measure of  convergence rate of a sequence, but this is a 
sequence that we might say converges very quickly. 
\end{solution}
\end{questions}
\end{document}
