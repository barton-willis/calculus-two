\documentclass[12pt,fleqn]{exam}
\usepackage{amssymb}
\usepackage[intlimits]{amsmath}
\usepackage{epsfig}
\usepackage{upgreek}
\usepackage[super]{nth}
\usepackage[colorlinks=true,linkcolor=black,anchorcolor=black,citecolor=black,filecolor=black,menucolor=black,runcolor=black,urlcolor=black]{hyperref}
\usepackage[letterpaper, margin=0.75in]{geometry}
\addpoints
\boxedpoints
\pointsinmargin
\pointname{pts}
\usepackage{tikz}
\usepackage{tkz-euclide}
\usetikzlibrary{shapes.geometric}
\usetikzlibrary{calc}
\usepackage[final]{microtype}
\frenchspacing
\usepackage[american]{babel}
\usepackage[T1]{fontenc}
\usepackage[]{fourier}
\usepackage{isomath}
\usepackage{upgreek,amsmath}
\usepackage{graphicx}

\newcommand{\dotprod}{\, {\scriptzcriptztyle\stackrel{\bullet}{{}}}\,}

\newcommand{\reals}{\mathbf{R}}
\newcommand{\lub}{\mathrm{lub}} 
\newcommand{\glb}{\mathrm{glb}} 
\newcommand{\complex}{\mathbf{C}}
\newcommand{\dom}{\mbox{dom}}
\newcommand{\range}{\mbox{range}}
\newcommand{\cover}{{\mathcal C}}
\newcommand{\integers}{\mathbf{Z}}
\newcommand{\vi}{\, \mathbf{i}}
\newcommand{\vj}{\, \mathbf{j}}
\newcommand{\vk}{\, \mathbf{k}}
\newcommand{\bi}{\, \mathbf{i}}
\newcommand{\bj}{\, \mathbf{j}}
\newcommand{\bk}{\, \mathbf{k}}
\DeclareMathOperator{\Arg}{\mathrm{Arg}}
\DeclareMathOperator{\Ln}{\mathrm{Ln}}
\newcommand{\imag}{\, \mathrm{i}}

\usepackage{graphicx}
\usepackage{color}
%\shadedsolutions
%\definecolor{SolutionColor}{rgb}{1,0.72,0.46} %{0.8,0.9,1}
\newcommand\AM{\textsc{am}}
\newcommand\PM{\textsc{pm}}
     
\newcommand{\quiz}{14}
\newcommand{\term}{Fall}
\newcommand{\due}{Thursday 12 October 13:20}
\newcommand{\class}{MATH 202, Fall \the\year}
\begin{document}
\large
\vspace{0.1in}
\noindent\makebox[3.0truein][l]{\textbf{\class}}
\textbf{Name:} \hrulefill \\
\noindent \makebox[3.0truein][l]{\textbf{In class work  \quiz}}
\textbf{Row and Seat}:\hrulefill\\

\noindent \emph{“You are never dedicated to something you have complete confidence in.''} \hfill 
   \\  $\phantom{xxx}$ \hfill {\sc Robert M. Pirsig}


\noindent  In class work  \textbf{\quiz}  has questions \textbf{1} 
through  \textbf{\numquestions} \/ with a total of 
\textbf{\numpoints\/}  points.Turn in your work at the end of class 
\emph{on paper}. This assignment is due \emph{\due}.

\vspace{0.1in}


\begin{questions} 
    
   \question[2] When Morwenna graduates from UNK and starts her first job, she expects to earn a 
   starting annual salary of \$42,000. She plans to work for 42 years
   and she expects to earn a 3\% raise each year. Thus, in her 
   $\mathrm{n}^{\mathrm{th}}$  year 
   of work, her salary is $42,000 \times 1.03^{n-1}$. During 
   Morwenna's 42 years of labor, how much will she earn?

   \begin{solution}[2.5in]
   
    
   \end{solution}

   \question Given a formula for a sequence $b$, find its limit.
   Show all of your work.

   \begin{parts}

    \part [2] $\displaystyle b_n = \sum_{k=0}^{n} \left(\frac{2}{3}\right)^k$.
    \begin{solution}%[2.5in]
   
    
    \end{solution}

    \newpage 
    \part [2] $\displaystyle b_n = \sum_{k=0}^{n} \left(\frac{3}{2}\right)^k$.
    \begin{solution}%[2.5in]
   
    
    \end{solution}






   \end{parts}
 
   \newpage

   \question [2] The Newton–Raphson method\footnote{Raphson invented
   the method before Newton. If you didn't learn the Newton–Raphson method
   in Calculus I, I should tell the Office of Student Records to 
   expunge your MATH 115 credit.} is a way to find an 
   approximate solution to an equation $F(x)=0$. Specifically,
   the method starts with a guess for the solution, call it $a_1$,
   and then refines the guess with values $a_2, a_3, \dots$. 
   This sequence is called a \emph{Newton sequence}.
   If all goes well, the sequence $a$ converges to 
   solution to $F(x)=0$. Specifically for $F(x) = x^2 - 2$ and
   an initial guess of $1$, the Newton sequence  is defined
   recursively by 
   \begin{equation*}
    a_{n+1} = \begin{cases} 1  & n=0  \\
                            a_n - \frac{a_n^2 - 2}{2 a_n} & n > 0
    \end{cases}.
\end{equation*}
Assuming that the sequence $a$ converges to a positive number, find the numerical
value of $\displaystyle \lim_{n \to \infty} a_n$.  Use the 
fact that if $\displaystyle \lim_{n \to \infty} a_n  = L$,
then $\displaystyle \lim_{n \to \infty} a_{n+1} = L$.

\quad Many of you will have the urge to ``simplify'' $a_n - \frac{a_n^2 - 2}{2 a_n}$
to $\frac{a_n^2 + 2}{2 a_n}$ or possibly to 
$\frac{a_n}{2} + \frac{1}{a_n}$. Doing so is an OK thing to do,
but I suggest doing a bit more `T' from GNAT\footnote{GNAT = Graphical,
Numerical, Algebraic, Think. I possibly invented the acronym, but 
Deborah Hughes Hallett invented, or at least popularized, the concept.} before you give into your urge to simplify.


\end{questions}
\end{document}
