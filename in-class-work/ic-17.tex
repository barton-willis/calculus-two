\documentclass[12pt,fleqn,answers]{exam}
\usepackage{amssymb}
\usepackage[intlimits]{amsmath}
\usepackage{epsfig}
\usepackage{upgreek}
\usepackage[super]{nth}
\usepackage[colorlinks=true,linkcolor=black,anchorcolor=black,citecolor=black,filecolor=black,menucolor=black,runcolor=black,urlcolor=black]{hyperref}
\usepackage[letterpaper, margin=0.75in]{geometry}
\addpoints
\boxedpoints
\pointsinmargin
\pointname{pts}
\usepackage{tikz}
\usepackage{tkz-euclide}
\usetikzlibrary{shapes.geometric}
\usetikzlibrary{calc}
\usepackage[final]{microtype}
\frenchspacing
\usepackage[american]{babel}
\usepackage[T1]{fontenc}
\usepackage[]{fourier}
\usepackage{isomath}
\usepackage{upgreek,amsmath}
\usepackage{graphicx}

\newcommand{\dotprod}{\, {\scriptzcriptztyle\stackrel{\bullet}{{}}}\,}

\newcommand{\reals}{\mathbf{R}}
\newcommand{\lub}{\mathrm{lub}} 
\newcommand{\glb}{\mathrm{glb}} 
\newcommand{\complex}{\mathbf{C}}
\newcommand{\dom}{\mbox{dom}}
\newcommand{\range}{\mbox{range}}
\newcommand{\cover}{{\mathcal C}}
\newcommand{\integers}{\mathbf{Z}}
\newcommand{\vi}{\, \mathbf{i}}
\newcommand{\vj}{\, \mathbf{j}}
\newcommand{\vk}{\, \mathbf{k}}
\newcommand{\bi}{\, \mathbf{i}}
\newcommand{\bj}{\, \mathbf{j}}
\newcommand{\bk}{\, \mathbf{k}}
\DeclareMathOperator{\Arg}{\mathrm{Arg}}
\DeclareMathOperator{\Ln}{\mathrm{Ln}}
\newcommand{\imag}{\, \mathrm{i}}

\usepackage{graphicx}
\usepackage{color}
%\shadedsolutions
%\definecolor{SolutionColor}{rgb}{1,0.72,0.46} %{0.8,0.9,1}
\newcommand\AM{\textsc{am}}
\newcommand\PM{\textsc{pm}}
     
\newcommand{\quiz}{17}
\newcommand{\term}{Fall}
\newcommand{\due}{Tuesday 26 October 13:20}
\newcommand{\class}{MATH 202, Fall \the\year}
\begin{document}
%\large
\vspace{0.1in}
\noindent\makebox[3.0truein][l]{\textbf{\class}}
\textbf{Name:} \hrulefill \\
\noindent \makebox[3.0truein][l]{\textbf{In class work  \quiz}}
\textbf{Row and Seat}:\hrulefill\\

\noindent \emph{“The more I read, the more I acquire, the more certain I am that I know nothing.”} \hfill {\sc Voltaire }

\vspace{0.1in}

\noindent  In class work  \textbf{\quiz}  has questions \textbf{1} 
through  \textbf{\numquestions} \/ with a total of 
\textbf{\numpoints\/} points. Turn in your work at the end of class 
\emph{on paper}. This assignment is due \emph{\due}.

\vspace{0.1in}



\begin{questions} 

\question [1] Find the numerical value of $\displaystyle \lim_{x \to \infty}   \left(1 + \frac{1}{x}\right)^x$.  Careful: This is 
an indeterminate form of the type $1^\infty$.  To start, I suggest that you use the technique
$
  \displaystyle  \lim_{x \to \infty}   \left(1 + \frac{1}{x}\right)^x = \lim_{x \to \infty}  \mathrm{e}^{x   \ln (1 + \frac{1}{x})}.
$

\begin{solution}[2.6in]
\begin{align*}
  \lim_{x \to \infty}  \mathrm{e}^{x   \ln (1 + \frac{1}{x})} &=  \lim_{x \to \infty} 
                      \frac{\ln(1+\frac{1}{x})} {\frac{1}{x}}, \\
\intertext{This is an indeterminate form of the type $\frac{0}{0}$, so let's try the l'H\^opital rule.}
                                      &=  \lim_{x \to \infty}  \frac{\frac{1}{1 + \frac{1}{x}} \left(- \frac{1}{x^2} \right)} {-\frac{1}{x^2}}, \\
                                      &= 1.
\end{align*}
So $\displaystyle  \lim_{x \to \infty}   \left(1 + \frac{1}{x}\right)^x = \mathrm{e}$.
\end{solution}

\question [1] Use the \emph{ratio} test to determine of the series $\sum_{k=0}^\infty \frac{  \left(\frac{k}{3} \right)^k }{ k!}$ converges or diverges.
\begin{solution}%[2.6in]
Algebra is supposed to be easier than calculus, so let's start by simplifying
\begin{equation*}
\frac{ (\frac{k+1}{3})^{k+1}}{(k+1)!} \times \frac{k!}{ (\frac{k}{3})^k} = 
\frac{1}{3} (\frac{k+1}{k})^k = \frac{1}{3}  (1 + \frac{1}{k})^k.
\end{equation*}
So $\displaystyle  \lim_{x \to \infty} \frac{ (\frac{k+1}{3})^{k+1}}{(k+1)!} \times \frac{k!}{ (\frac{k}{3})^k} = \frac{1}{3}$.
And so $\displaystyle \sum_{k=0}^\infty \frac{  \left(\frac{k}{3} \right)^k }{ k!}$ converges. 
\end{solution}

\newpage

\question [1] Define a sequence $s$ by $\displaystyle s_n = \sum_{k=1}^n \frac{(-1)^{k+1}}{\sqrt{k}}$.  This is a convergent alternating series.
Also define \mbox{$\displaystyle s_\infty = \lim_{n \to \infty}  s_n$.}

\begin{parts}

\part [1] Use Desmos to graph $s$ on the interval $[1, 2, \dots, 150]$. Also use Desmos to find the numeric values of $s_{149}$ and  $s_{150}$.
As best you can, reproduce a cartoon of the graph of $s$.

\begin{solution}[1.6in]
\end{solution}

\part[1] From the theory of convergent alternating series, we know that $s_{150} < s_\infty < s_{149}$.   Looking at the graph of $s$,
I would guess that $s_\infty$ is pretty close to the arithmetic average of $s_{150} $ and $s_{149}$; that is $s_\infty  \approx \frac{s_{150} + s_{149}}{2}$.  Find the numeric value of $\frac{s_{150} + s_{149}}{2}$.

\begin{solution}[0.6in]
\end{solution}

\part [1]  Define a sequence $w$ by $w_n = \frac{s_{n+1} + s_n}{2}$.  With a bit of effort, we could prove that the sequence $w$ is a convergent
alternating sequence that converges to $s_\infty$.  Use Desmos to graph the sequences $s$ and  $w$ on the interval $[1, 2, \dots, 150]$.
Which sequence would you say converges ``faster''?    As best you can, reproduce a cartoon graphs of $s$ and $w$.

\end{parts}


\end{questions}
\end{document}
