\documentclass[12pt,fleqn]{exam}
\usepackage{amssymb}
\usepackage[intlimits]{amsmath}
\usepackage{epsfig}
\usepackage{upgreek}
\usepackage[super]{nth}
\usepackage[colorlinks=true,linkcolor=black,anchorcolor=black,citecolor=black,filecolor=black,menucolor=black,runcolor=black,urlcolor=black]{hyperref}
\usepackage[letterpaper, margin=0.75in]{geometry}
\addpoints
\boxedpoints
\pointsinmargin
\pointname{pts}
\usepackage{tikz}
\usepackage{tkz-euclide}
\usetikzlibrary{shapes.geometric}
\usetikzlibrary{calc}
\usepackage[final]{microtype}
\frenchspacing
\usepackage[american]{babel}
\usepackage[T1]{fontenc}
\usepackage[]{fourier}

\usepackage{isomath}
\usepackage{upgreek,amsmath}
\usepackage{graphicx}

\newcommand{\dotprod}{\, {\scriptzcriptztyle\stackrel{\bullet}{{}}}\,}

\newcommand{\reals}{\mathbf{R}}
\newcommand{\lub}{\mathrm{lub}} 
\newcommand{\glb}{\mathrm{glb}} 
\newcommand{\complex}{\mathbf{C}}
\newcommand{\dom}{\mbox{dom}}
\newcommand{\range}{\mbox{range}}
\newcommand{\cover}{{\mathcal C}}
\newcommand{\integers}{\mathbf{Z}}
\newcommand{\vi}{\, \mathbf{i}}
\newcommand{\vj}{\, \mathbf{j}}
\newcommand{\vk}{\, \mathbf{k}}
\newcommand{\bi}{\, \mathbf{i}}
\newcommand{\bj}{\, \mathbf{j}}
\newcommand{\bk}{\, \mathbf{k}}
\DeclareMathOperator{\Arg}{\mathrm{Arg}}
\DeclareMathOperator{\Ln}{\mathrm{Ln}}
\newcommand{\imag}{\, \mathrm{i}}
\newcommand{\erf}{\mathrm{erf}}
\newcommand{\e}{\mathrm{e}}

\usepackage{graphicx}
\usepackage{color}
%\shadedsolutions
%\definecolor{SolutionColor}{rgb}{1,0.72,0.46} %{0.8,0.9,1}
\newcommand\AM{\textsc{am}}
\newcommand\PM{\textsc{pm}}


     
\newcommand{\quiz}{21}
\newcommand{\term}{Fall}
\newcommand{\due}{Thursday 9 November 13:20}
\newcommand{\class}{MATH 202, Fall \the\year}
\begin{document}
\large
\noindent\makebox[3.0truein][l]{\textbf{\class}}
\textbf{Name:} \hrulefill \\
\noindent \makebox[3.0truein][l]{\textbf{In class work  \quiz}}
\textbf{Row and Seat}:\hrulefill\\



\noindent  In class work  \textbf{\quiz}  has questions \textbf{1} 
through  \textbf{\numquestions} \/ with a total of 
\textbf{\numpoints\/} points. Turn in your work at the end of class 
\emph{on paper}. This assignment is due at \emph{\due}.

\vspace{0.1in}

\noindent \emph{“The place to improve the world is first in one's 
own heart and head and hands, and then work outward from there.”}
 \phantom{xxx} \hfill {\sc  Robert  Pirsig}


\begin{questions} 



\question For all real numbers $x$, we have
$\displaystyle
   \sin(x) = \sum_{k=0}^\infty \frac{(-1)^k}{(2k+1)!} x^{2 k + 1}.
$

\begin{parts}

    \part [2] Find the power series representation for $\sin(x) - x$
    centered at zero. \textbf{Hint:} When you don't know where to start, go to your
    happy place: write the first few terms of the Taylor series for sine
    centered at zero. Then subtract $x$.

    \begin{solution}[3.05in]

    \end{solution}


    \part[2]  For $x \neq 0$, find the \emph{first two nonzero terms} 
    in a power series representation for $\frac{\sin(x) - x}{x^3}$.
    Again, try visiting your happy place.
    \begin{solution}%[2.5in]

    \end{solution}

\newpage
    \part [2] Use the above result to find the \emph{numerical value}
     of the
    limit 
    \begin{equation*}
        \lim_{x \to 0} \frac{\sin(x) - x}{x^3}.
    \end{equation*}

\end{parts}
\end{questions}
\end{document}

\question [2] Use the facts for all real $x$ and $y$
\begin{align*}
  \e^{\imag x} &= \cos(x) + \imag \sin(x), \\
  \e^{\imag x}  \e^{\imag y}  &=   \e^{\imag (x + y)}
\end{align*}
to show that
\begin{align*}
    \cos{\left( x+y\right) }&=\cos{(x)} \cos{(y)}-\sin{(x)} \sin{(y)}, \\
    \sin{\left( x+y\right) }&=\cos{(x)} \sin{(y)}+\sin{(x)} \cos{(y)}
\end{align*}
are both identities.  Here is a start:
\begin{equation*}
    \cos(x+y) + \imag \sin(x+y) = 
    e^{\imag (x+y)} = e^{\imag x} e^{\imag y}
        = (\cos(x) + \imag \sin(x)) (\cos(y) + \imag \sin(y)).
\end{equation*}

\begin{solution}%[2.5in]

\end{solution}
    
\newpage
