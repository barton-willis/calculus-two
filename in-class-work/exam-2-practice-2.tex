\documentclass[12pt]{exam}

\usepackage{pifont}
\usepackage{dingbat}
\usepackage{amsmath}
\usepackage{fleqn}
\usepackage{epsfig,upgreek}
\usepackage{mathptm}
\newcommand{\reals}{\mathbf{R}}
\addpoints
\boxedpoints
\pointsinmargin
\pointname{pts}

\newcommand{\dotprod}{\, {\scriptzcriptztyle \stackrel{\bullet}{{}}}\,}
\begin{document}
\large
\vspace{0.1in}
\noindent\makebox[3.0truein][l]{{\bf MATH 202, Fall \the\year}}
{\bf Name:}\hrulefill\
\noindent \makebox[3.0truein][l]{\bf Practice Exam}
{\bf Row and Seat:}\hrulefill\
%\normalsize



\begin{questions}

\question Find the value of each indefinite or definite integral.
\begin{parts}

\part \(\displaystyle\int x e^{x^2} \, \mathrm{d} x = \)

\begin{solution}[2.0in]
\end{solution}
\part \(\displaystyle\int_0^1 \frac{x}{\left(1+x^2\right)^{3/2}} \, \mathrm{d}x = \)
\begin{solution}[2.0in]
\end{solution}
\part \(\displaystyle\int x \sqrt{1-x^2} \, \mathrm{d}x = \)
\begin{solution}[2.0in]
\end{solution}

\part \(\displaystyle \int \tan^{-1}(x) \, \mathrm{d}x = \)
\begin{solution}[2.0in]
\end{solution}
\part \(\displaystyle \int x \ln(x) \, \mathrm{d}x = \)
\begin{solution}[2.0in]
\end{solution}
\part \(\displaystyle \int_0^1 x e^{-x} \, \mathrm{d}x = \)
\begin{solution}[2.0in]
\end{solution}
\part \(\displaystyle \int \frac{1}{(x+5)(x+9)} \, \mathrm{d}x = \)
\begin{solution}[2.0in]
\end{solution}
\part \(\displaystyle \int \frac{x^2}{(x+5)(x+9)} \, \mathrm{d}x = \)
\begin{solution}[2.0in]
\end{solution}
\part \(\displaystyle \int \cos^2(x) \, \mathrm{d}x = \)
\begin{solution}[2.0in]
\end{solution}
\part \(\displaystyle \int \cos^3(x) \sin(x) \, \mathrm{d}x = \)

\end{parts}


\question Find the numerical value of each improper integral.

\begin{parts}

\part $\int_0^\infty x \mathrm{e}^{-x^2} \, \mathrm{d} x$
\begin{solution}[2.0in]
\end{solution}
\part $\int_0^\infty x \mathrm{e}^{-x} \, \mathrm{d} x$
\begin{solution}[2.0in]
\end{solution}
\part $\int_{-\infty}^\infty   \frac{1}{x^2 + 9} \, \mathrm{d} x$
\begin{solution}[2.0in]
\end{solution}
\part $\int_0^1 \frac{1}{x^{9/10}} \, \mathrm{d} x$
\end{parts}
\begin{solution}[2.0in]
\end{solution}
\question[2] When Morwenna graduates from UNK and starts her first job, she expects to earn a 
   starting annual salary of \$42,000. She plans to work for 42 years
   and she expects to earn a 3\% raise each year. Thus, in her 
   $\mathrm{n}^{\mathrm{th}}$  year 
   of work, her salary is $42,000 \times 1.03^{n-1}$. During 
   Morwenna's 42 years of labor, how much will she earn?

   \begin{solution}[2.5in]
   
    
   \end{solution}

   \question Given a formula for a sequence $b$, find its limit.
   Show all of your work.

   \begin{parts}

    \part [2] $\displaystyle b_n = \sum_{k=0}^{n} \left(\frac{2}{3}\right)^k$.
    \begin{solution}%[2.5in]
   
    
    \end{solution}

    \newpage 
    \part [2] $\displaystyle b_n = \sum_{k=0}^{n} \left(\frac{3}{2}\right)^k$.
    \begin{solution}%[2.5in]
   
        \end{solution}
    \end{parts}
    




  \question  Show that the sequence whose formula is 
  $a_k = \sqrt{k^2+ 3 k + 1} - k$ converges. Show all of your work.
  \begin{solution}%[2.5in]
  We'll use the trick of moving the radicals to the denominator. We have
  \begin{align*}
  \lim_{k \to \infty} \left(\sqrt{k^2+1} - k \right)  &= \lim_{k \to \infty} \left(\sqrt{k^2+1} - k \right)   \times \frac{\sqrt{k^2+1}  + k}{\sqrt{k^2+1}  + k}, \\
                                                                 &= \lim_{k \to \infty} \frac{(k^2+1) - k^2}{\sqrt{k^2+1}  + k}, \\
                                                                 &=  \lim_{k \to \infty} \frac{1}{\sqrt{k^2+1}  + k}, \\
      \intertext{We have $\displaystyle \lim_{x \to \infty}   \left( \sqrt{k^2+1}  + k \right) = \infty$, so we have}                                                        
                                                                 &=  0
  \end{align*}
  Although the calculation $\displaystyle \lim_{k \to \infty} \frac{1}{\sqrt{k^2+1}  + k} = \frac{1}{\infty} = 0$ is based on a theorem, it's not   
  something you should do in public. Specifically, the theorem is: 
  If $\displaystyle \lim_a   F \in \reals$ and $\displaystyle \lim_a G = \infty$,
  then $\displaystyle \lim_a \frac{F}{G} = 0$. Be careful with this--the limit of the numerator must be a real number.
  \end{solution}

  \newpage
  \question [2] Determine if the sequence whose formula is 
  $b_k = k \ln \left(1+\frac{8}{k} \right)$ converges. If it does, find its 
  limit. As always, show your work.
  \begin{solution}%[2.5in]
We need the limit of a product. Since $\displaystyle \lim_{k \to \infty} k = \infty$ and $\displaystyle \lim_{k \to \infty} \ln \left(1+\frac{8}{k} \right) = 0$, we have an indeterminate form of the type $0 \times \infty$. By rearranging this to $\frac{\ln\left(1+\frac{8}{k} \right) }{\frac{1}{k}}$,
we have an  indeterminate form of the type $\frac{0}{0}$. And maybe the l'Hôpital rule will help.  We have
\begin{align*}
  \lim_{ k  \to \infty} \ln \left(1+\frac{8}{k} \right) &=  \lim_{k \to \infty} \frac{ \ln \left(1+\frac{8}{k} \right) }{\frac{1}{k}}, \\
                                                                                          &= \lim_{k \to \infty} \frac{ -\frac{8}{\left( \frac{8}{k}+1\right) \, {{k}^{2}}}}{-\frac{1}{k^2}}\\
                                                                                          &=  \lim_{k \to \infty} \frac{8}{\frac{8}{k}+1}, \\
                                                                                          &= 8.
\end{align*}
  Actually, until we've determined that the limit of the quotient of derivatives is a real number, equality between the first and second lines is 
  not certain.  Notionally, we could use $\overset{?}{=}$ To mean equality as long as the right side is a real number, but that notation (I didn't invent it) is nonstandard.
  \end{solution}

  \newpage
  \question A sequence $c$ is defined recursively by
  \begin{equation*}
      c_n = \begin{cases} 2 & n=0 \\
                          5 & n=1 \\
                          5 c_{n-1}-6 c_{n-2} & n=2,3,4,\dots
      \end{cases}
    \end{equation*}
        
\begin{parts}
\part[2] Find the numeric values of $c_2, c_3$, and $c_4$.
\begin{solution}[2.5in]
\begin{align*}
   c_2 &= 5 c_1 - 6 c_0 = 5 \times  5 - 6 \times 2 = 25-12 = 13,\\
   c_3 &= 5 c_2 - 6 c_1 = 5 \times  13 - 6 \times 5 = 65-30 = 35,\\
    c_4 &= 5 c_3 - 6 c_2 = 5 \times  35 - 6 \times 13= 25-12 = 97.
\end{align*}
\end{solution}






   \end{parts}
   
  \question[2] Find the \emph{numeric value} of the integral 
  $\int_0^\infty \frac{x}{1+x^4} \, \mathrm{d} x$.
  \textbf{Hint:} To find an antiderivative of $\int \frac{x}{1+x^4} \, \mathrm{d} x$, use 
  the substitution $z = x^2$.

  \begin{solution}[3.0in] Let's begin by finding an antiderivative; once
    we found it, we'll use the FTC along with a limit to find the value 
    of the improper integral. We have
    \begin{align*}
    \int \frac{x}{1+x^4} \, \mathrm{d} x &= \frac{1}{2}  \int \, \frac{1}{1+(x^2)^2} \, \mathrm{d} x^2,  
                          && \left (x \mathrm{d} x = \frac{1}{2} \mathrm{d} x^2 \right), \\
                                                         &= \frac{1}{2}  \int  \frac{1}{1+z^2} \, \mathrm{d} z, 
                                                         && (\mbox{replace } x^2 \mbox{ by } z )\\
                                                         &= \frac{1}{2} \arctan(z), 
                                                         &&(\mbox{standard antiderivative})\\
                                                         &= \frac{1}{2} \arctan(x^2) && (\mbox{replace} z \mbox{ by } x^2 ).
    \end{align*}
    Second, we take on the improper integral:
     \begin{align*}
    \int_0^\infty  \frac{x}{1+x^4} \, \mathrm{d} x &= \lim_{a \to \infty}  \int_0^a  \frac{x}{1+x^4} \, \mathrm{d} x, \\
                &=  \lim_{a \to \infty} \left( \frac{1}{2} \arctan(x^2) \right |_{0}^a, \\
                &=  \lim_{a \to \infty} \left( \frac{1}{2} \arctan(a^2)  -  \frac{1}{2} \arctan(0) \right), \\
                 &=  \lim_{a \to \infty} \left( \frac{1}{2} \arctan(a^2)  \right), \\
                 &= \frac{\uppi}{4}
    \end{align*}
  \end{solution}


  \question [1] Show that $\int_0^\infty \frac{28 + \cos(x)}{1+x^2} \, \mathrm{d} x$
  converges. To do this, use a comparison test with 
  $ \frac{\alpha}{1+x^2}$, where $\alpha$ is a number that you cleverly 
  choose.
  \begin{solution} For all real numbers $x$, we have $ 27 \leq 28 + \cos(x) \leq 29$. Let's (cleverly) choose $\alpha$
  to be $29$. Then for all  real numbers $x$, we have
  \begin{equation}
    0 \leq  \frac{28 + \cos(x)}{1+x^2} \leq  \frac{29}{1+x^2}.
  \end{equation}
  But $\int_0^\infty \frac{29}{1+x^2} \, \mathrm{d} x$ converges, so $\int_0^\infty \frac{28 + \cos(x)}{1+x^2} \, \mathrm{d} x$ converges.
  
  \textbf{Be careful} We only know that $\int_0^\infty \frac{28 + \cos(x)}{1+x^2} \, \mathrm{d} x$ is a real number,
  but the comparison test \textbf{doesn't} tell us its value.  We'll it does tell us that
  \begin{equation*}
    \int_0^\infty \frac{28 + \cos(x)}{1+x^2} \, \mathrm{d} x  \leq  \int_0^\infty \frac{29}{1+x^2} \, \mathrm{d} x
    = \frac{29 \ensuremath{\pi} }{2} \approx 45.553093477052.
  \end{equation*}
  Numerical integration gives us the approximation $ \int_0^\infty \frac{28 + \cos(x)}{1+x^2} \, \mathrm{d} x \approx
   44.560$
  \end{solution}

  %\newpage
  \question [1] Show that 
  $\int_1^\infty \frac{107 + \mathrm{e}^{-x}}{1+x^2} \, \mathrm{d} x$
  converges. To do this, use a limit comparison test. 
  \begin{solution}[3.0in] We know that  $\int_1^\infty \frac{1}{1+x^2} \, \mathrm{d} x$ converges. And
  for all real $x \geq 1$ we have $\frac{107 + \mathrm{e}^{-x}}{1+x^2} > 0$ and $\frac{1}{1+x^2} >  0$.
  Finally, everything in sight is continuous; so look at
  \begin{align*}
   \lim_{x \to \infty} \frac{\frac{1}{1+x^2}} {\frac{107 + \mathrm{e}^{-x}} {1+x^2} } &=
    \lim_{x \to \infty} \frac{1}{107 + \mathrm{e}^{-x}}, \\
    &= \frac{1}{107}.   
    \end{align*}
  So $\int_1^\infty \frac{107 + \mathrm{e}^{-x}}{1+x^2} \, \mathrm{d} x$ converges.
  \end{solution}   
  
  
\question Use the integral test to show that the series $\sum_{k=0}^\infty \frac{1}{1+k^2} $ converges.
\end{questions}




\end{document}
