\documentclass[12pt,fleqn]{exam}
\usepackage{amssymb}
\usepackage[intlimits]{amsmath}
\usepackage{epsfig}
\usepackage{upgreek}
\usepackage[super]{nth}
\usepackage[colorlinks=true,linkcolor=black,anchorcolor=black,citecolor=black,filecolor=black,menucolor=black,runcolor=black,urlcolor=black]{hyperref}
\usepackage[letterpaper, margin=0.75in]{geometry}
\addpoints
\boxedpoints
\pointsinmargin
\pointname{pts}
\usepackage{tikz}
\usepackage{tkz-euclide}
\usetikzlibrary{shapes.geometric}
\usetikzlibrary{calc}
\usepackage[final]{microtype}
\usepackage[american]{babel}
\usepackage[T1]{fontenc}
\usepackage[]{fourier}

\usepackage{isomath}
\usepackage{upgreek,amsmath}
\usepackage{graphicx}

\newcommand{\dotprod}{\, {\scriptzcriptztyle\stackrel{\bullet}{{}}}\,}

\newcommand{\reals}{\mathbf{R}}
\newcommand{\lub}{\mathrm{lub}} 
\newcommand{\glb}{\mathrm{glb}} 
\newcommand{\complex}{\mathbf{C}}
\newcommand{\dom}{\mbox{dom}}
\newcommand{\range}{\mbox{range}}
\newcommand{\cover}{{\mathcal C}}
\newcommand{\integers}{\mathbf{Z}}
\newcommand{\vi}{\, \mathbf{i}}
\newcommand{\vj}{\, \mathbf{j}}
\newcommand{\vk}{\, \mathbf{k}}
\newcommand{\bi}{\, \mathbf{i}}
\newcommand{\bj}{\, \mathbf{j}}
\newcommand{\bk}{\, \mathbf{k}}
\DeclareMathOperator{\Arg}{\mathrm{Arg}}
\DeclareMathOperator{\Ln}{\mathrm{Ln}}
\newcommand{\imag}{\, \mathrm{i}}
\newcommand{\erf}{\mathrm{erf}}
\newcommand{\e}{\mathrm{e}}

\usepackage{graphicx}
\usepackage{color}
%\shadedsolutions
%\definecolor{SolutionColor}{rgb}{1,0.72,0.46} %{0.8,0.9,1}
\newcommand\AM{\textsc{am}}
\newcommand\PM{\textsc{pm}}


     
\newcommand{\quiz}{25}
\newcommand{\term}{Fall}
\newcommand{\due}{Thursday 30 November 13:20}
\newcommand{\class}{MATH 202, Fall \the\year}
\begin{document}
\large
\noindent\makebox[3.0truein][l]{\textbf{\class}}
\textbf{Name:} \hrulefill \\
\noindent \makebox[3.0truein][l]{\textbf{In class work  \quiz}}
\textbf{Row and Seat}:\hrulefill\\



\noindent  In class work  \textbf{\quiz}  has questions \textbf{1} 
through  \textbf{\numquestions} \/ with a total of 
\textbf{\numpoints\/} points. Turn in your work at the end of class 
\emph{on paper}. This assignment is due at \emph{\due}.

\vspace{0.1in}

\noindent \emph{“Time is never wasted if you remember to bring along something to read.”}
 \\ \phantom{xxx} \hfill {\sc Thomas Pynchon}


\begin{questions} 
    
  



\question In polar coordinates, an equation of a curve 
$\mathcal{C}$ is $r = \sqrt{\frac{1}{4} - \sin(\theta)^2}$.

\begin{parts}

\part[2] Use Desmos to draw a graph of this polar equation. As best you
can, reproduce the graph here.

\begin{solution}[2.5in]
    
\end{solution}

\part[2] Find all solutions to $0 = \sqrt{\frac{1}{4} - \sin(\theta)^2}$.
These solutions give all the points on the curve that intersect the origin
with $\theta \in [0, 2 \uppi]$. To find \emph{all} solutions to this
equation, use the \emph{source of all knowledge} (\textbf{SOAK}), that is, 
the unit circle.

\begin{solution}%[2.5in]
    
\end{solution}

\newpage

\part[2] For each intersection of $\mathcal{C}$ with the origin, find
the slope of the tangent line. Using Desmos, verify that 
you have found the correct tangent lines. \textbf{Note:} Desmos
refuses\footnote{I think Desmos should hire some UNK CS graduates 
to fix this.} to graph a polar curve of the form $\theta = f(r)$. 
And it particular, it will not graph the polar curve 
$\theta = \frac{\uppi}{4}$,  for example. To workaround this,
you'll need to find the cartesian equation of the tangent lines.

\end{parts}


\end{questions}

\vfill 
\noindent \textbf{Optional} For extra fun, find a cartesian equation of the curve $\mathcal{C}$.
Show that for $x \in [-\frac{1}{2}, \frac{1}{2}]$, a cartesian
equation of the curve is $y = \pm \frac{\sqrt{\sqrt{64 {{x}^{2}}+9}-8 {{x}^{2}}-3}}{{{2}^{\frac{3}{2}}}}$.
And show that the other two solutions are not real. Finally,
are there any values of $x$ that allow the nested radical 
$\sqrt{\sqrt{64 {{x}^{2}}+9}-8 {{x}^{2}}-3}$ to denest?
    
\end{document}
