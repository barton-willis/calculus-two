\documentclass[12pt,fleqn]{exam}
\usepackage{amssymb}
\usepackage[intlimits]{amsmath}
\usepackage{epsfig}
\usepackage{upgreek}
\usepackage[super]{nth}
\usepackage[colorlinks=true,linkcolor=black,anchorcolor=black,citecolor=black,filecolor=black,menucolor=black,runcolor=black,urlcolor=black]{hyperref}
\usepackage[letterpaper, margin=0.75in]{geometry}
\addpoints
\boxedpoints
\pointsinmargin
\pointname{pts}
\usepackage{tikz}
\usepackage{tkz-euclide}
\usetikzlibrary{shapes.geometric}
\usetikzlibrary{calc}
\usepackage[final]{microtype}
\frenchspacing
\usepackage[american]{babel}
\usepackage[T1]{fontenc}
\usepackage[]{fourier}
\usepackage{isomath}
\usepackage{upgreek,amsmath}
\usepackage{graphicx}

\newcommand{\dotprod}{\, {\scriptzcriptztyle\stackrel{\bullet}{{}}}\,}

\newcommand{\reals}{\mathbf{R}}
\newcommand{\lub}{\mathrm{lub}} 
\newcommand{\glb}{\mathrm{glb}} 
\newcommand{\complex}{\mathbf{C}}
\newcommand{\dom}{\mbox{dom}}
\newcommand{\range}{\mbox{range}}
\newcommand{\cover}{{\mathcal C}}
\newcommand{\integers}{\mathbf{Z}}
\newcommand{\vi}{\, \mathbf{i}}
\newcommand{\vj}{\, \mathbf{j}}
\newcommand{\vk}{\, \mathbf{k}}
\newcommand{\bi}{\, \mathbf{i}}
\newcommand{\bj}{\, \mathbf{j}}
\newcommand{\bk}{\, \mathbf{k}}
\DeclareMathOperator{\Arg}{\mathrm{Arg}}
\DeclareMathOperator{\Ln}{\mathrm{Ln}}
\newcommand{\imag}{\, \mathrm{i}}

\usepackage{graphicx}
\usepackage{color}
%\shadedsolutions
%\definecolor{SolutionColor}{rgb}{1,0.72,0.46} %{0.8,0.9,1}
\newcommand\AM{\textsc{am}}
\newcommand\PM{\textsc{pm}}
     
\newcommand{\quiz}{11}
\newcommand{\term}{Fall}
\newcommand{\due}{Tuesday 3 October 13:20}
\newcommand{\class}{MATH 202, Fall \the\year}
\begin{document}
\large
\vspace{0.1in}
\noindent\makebox[3.0truein][l]{\textbf{\class}}
\textbf{Name:} \hrulefill \\
\noindent \makebox[3.0truein][l]{\textbf{In class work  \quiz}}
\textbf{Row and Seat}:\hrulefill\\

\begin{quote}
    \emph{``Singularity is almost invariably a clue.''} \hfill 
    {\sc Sherlock Holmes}
\end{quote}

\noindent  In class work  \textbf{\quiz}  has questions \textbf{1} 
through  \textbf{\numquestions} \/ with a total of 
\textbf{\numpoints\/}  points.Turn in your work at the end of class 
\emph{on paper}. This assignment is due \emph{\due}.

\vspace{0.1in}


\begin{questions} 
    
    \question These questions involve the region $Q$ defined 
    by 
    \begin{equation*}
        Q = \{(x,y) \mid 0 \leq y \leq \exp(-x),  0 \leq x < \infty \}.
    \end{equation*}
    For each of the following, you will need to evaluate an improper integral
    of the form $\int_0^\infty F(x) \, \mathrm{d} x$. You will need 
    to evaluate such integrals using
    \mbox{$\displaystyle
        \lim_{a \to \infty}  \int_0^a F(x) \, \mathrm{d} x.
    $}

\begin{parts}

    \part [1] Sketch the region $Q$. Make a pretty good guess at the 
    location of the \emph{centroid} of $Q$.
    \begin{solution}[1.5in]
    \end{solution} 

    \part [1] Find the \emph{area} of the region $Q$.

     \begin{solution}%[2.5in]
     \end{solution} 

     \newpage 
     \part [1] Find the $x$ coordinate of the \emph{centroid} of the 
     region $Q$.
     \begin{solution}[4.5in]
     \end{solution} 
   
     \part [1] Find the $y$ coordinate of the \emph{centroid} of the 
     region $Q$.
     \begin{solution}[2.5in]
     \end{solution} 

     \begin{solution}%[2.5in]
     \end{solution} 

     
\end{parts}

\newpage
 \question The floor function rounds a real number $x$ down to the 
 next integer that is less than or equal to $x$. For example, 
 $\lfloor \uppi \rfloor = 3$ and $\lfloor 3 \rfloor = 3$. The 
 1987 (Edition B) of \emph{Larry's Obscure Table of Obscure but 
 Useful Integrals}, lists the antiderivative
 (reprinted here with permission)
 $    \int  \lfloor x \rfloor  \, \mathrm{d} x = 
     \frac{1}{2} (2x -1) \lfloor x \rfloor - \frac{1}{2} \lfloor x \rfloor^2$.
  
\begin{parts}

    \part [1] Use Desmos to graph $\frac{1}{2} (2x -1) \lfloor x \rfloor - \frac{1}{2} \lfloor x \rfloor^2$.
    for $0 \leq x \leq 4$. Reproduce your graph here. 
     Does the graph appear to be continuous? (When an antiderivative 
     exists, it must be continuous.)

    \begin{solution}[2.5in]
    \end{solution} 

    
    \part [1] Evaluate the definite integral $\int_0^{\sqrt{42}}
     2 x \lfloor x^2+1 \rfloor \, \mathrm{d} x$.

\end{parts}
\end{questions}
\end{document}

