\documentclass[12pt,fleqn,answers]{exam}
\usepackage{pifont}
\usepackage{dingbat}
\usepackage{amssymb}
\usepackage{epsfig}
\usepackage{graphicx}
\usepackage[]{hyperref}
\usepackage{geometry}
\usepackage[intlimits]{amsmath}
\geometry{letterpaper, margin=0.75in}
\addpoints
\boxedpoints
\pointsinmargin
\pointname{pts}

\usepackage[activate={true,nocompatibility},final,tracking=true,kerning=true,factor=1100,stretch=10,shrink=10]{microtype}
\usepackage[american]{babel}
%\usepackage[T1]{fontenc}
\usepackage{fourier}
\usepackage{isomath}
\usepackage{upgreek,amsmath}
\usepackage{amssymb}

\newcommand{\dotprod}{\, {\scriptzcriptztyle
    \stackrel{\bullet}{{}}}\,}

\newcommand{\reals}{\mathbf{R}}
\newcommand{\lub}{\mathrm{lub}} 
\newcommand{\glb}{\mathrm{glb}} 
\newcommand{\complex}{\mathbf{C}}
\newcommand{\dom}{\mbox{dom}}
\newcommand{\cover}{{\mathcal C}}
\newcommand{\integers}{\mathbf{Z}}
\newcommand{\vi}{\, \mathbf{i}}
\newcommand{\vj}{\, \mathbf{j}}
\newcommand{\vk}{\, \mathbf{k}}
\newcommand{\bi}{\, \mathbf{i}}
\newcommand{\bj}{\, \mathbf{j}}
\newcommand{\bk}{\, \mathbf{k}}
\DeclareMathOperator{\Arg}{\mathrm{Arg}}
\DeclareMathOperator{\Ln}{\mathrm{Ln}}
\newcommand{\imag}{\, \mathrm{i}}
\newcommand{\range}{\mathrm{range}}
\newcommand{\ball}{\mathrm{ball}}
\newcommand{\LP}{\mathrm{LP}}

\usepackage{graphicx}
\newcommand\AM{{\sc am}}
\newcommand\PM{{\sc pm}}
     
\newcommand{\quiz}{0}
\newcommand{\term}{Fall}
\newcommand{\due}{Friday 25 August  at 11:59 \PM}
\begin{document}
\large
\vspace{0.1in}
\noindent\makebox[3.0truein][l]{{\bf MATH 202}}
{\bf Name:} \hrulefill \\
\noindent \makebox[3.0truein][l]{\bf Calculus Practice IV, \term \/ \the\year}
%{\bf Row:}\hrulefill\
\vspace{0.1in}

\noindent Here is an opportunity for you to maintain your calculus skills
over the summer. If you complete these problems,
digitize your work, and submit your work to Canvas, I will send you my
solutions. If you need some help with these questions, 
email me with your questions (\href{mailto:willisb@unk.edu}{willisb@unk.edu})

Completing this work is optional, and it does not 
enter into your class grade in any way--this work is 
 not a bonus, extra credit, or anything like that.

\begin{questions}


\question The graph in Figure 1 shows the graph of a wild and 
crazy function (the red curve) whose domain is $[-4,4]$ that
we'll unimaginatively call $F$. Define a function $G$ by
\begin{equation*}
   G(x) = \int_{-4}^x F(s) \, \mathrm{d} s.
\end{equation*}

As best you can, draw a graph of $G$.

\begin{figure}[h]
\begin{center}
\includegraphics[scale=0.4]{desmos-graph(49).png}
\end{center}
\caption{Graph of some wild and crazy function.}
\end{figure}
\end{questions}

\begin{solution}

To start, let's review the properties of a function $G$ that is defined by
\begin{equation*}
    G(x) = \int_{a}^x F(s) \, \mathrm{d}s,
\end{equation*} 
where $F$ is integrable on an interval $[a,b]$. We have:

\begin{itemize}
\item $G(a)=0$
\item $G$ is continuous on $[a,b]$.
\item $G$ is differentiable everywhere that $F$ is continuous.
\item Wherever $F$ is continuous, we have $G^\prime(x) = F(x)$.
\end{itemize}
Specifically for our function $F$ given graphically in Figure 1, we 
conclude that (b) $G(-4) = 0$, (b) $G$ is continuous on $[-4,4]$  
  (c)$G^\prime(x) = F(x)$  on $[-4,4]$. Further, since 
 $F(x) > 0$, we see 
  that $G^\prime(x) > 0$. This means that $G$ is an increasing 
  function. The steepest part of the graph of $G$ is at $0$, where
  $G^\prime(0) = 5$.  Finally, the derivative of $G$ is constant on 
  the interval $[-4,-1]$, so $G$ is a linear function on this interval.
  The same is true on the interval $[1,4]$.

  Putting all this together, a graph of $G$ looks something like

  \vfill 
  \newpage 
    \begin{center}
    \includegraphics[scale=0.4]{desmos-graph(51).png}
  \end{center}


  
\end{solution}
    

\end{document}